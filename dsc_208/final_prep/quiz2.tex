\documentclass{article}
\usepackage{amsmath}
\usepackage{amssymb}
\usepackage{amsthm}
\usepackage{enumitem}

\title{Quiz 2: DSC 208 Data Management for Analytics}
\author{}
\date{}

\begin{document}

\maketitle

\section*{Questions and Explanations}

\begin{enumerate}[label=\textbf{Question \arabic*.}]

\item If the primary key of a relation consists of the attributes A;B, then no record can have A = B.
    \begin{enumerate}[label=\alph*)]
        \item True
        \item False
    \end{enumerate}
    \textbf{Answer: b) False}
    \begin{itemize}
        \item \textit{Explanation:} A primary key consisting of attributes (A, B) means that the *combination* of values for A and B must be unique for each record. It does not mean that the individual values of A and B cannot be equal within a record. For example, if (A,B) is the primary key, a tuple (5,5) is perfectly valid, as long as there isn't another tuple (5,5) in the table. The constraint is on the uniqueness of the *pair*, not on the individual values within the pair.
    \end{itemize}

\item The table Arc(x,y) currently has the following tuples (note there are duplicates): (1,2), (1,2), (2,3), (3,4), (3,4), (4,1), (4,1), (4,1), (4,2). Compute the result of the query: Which of the following tuples is in the result?
    SQL query:
    \begin{verbatim}
    SELECT a1.x, a2.y, COUNT(*)
    FROM Arc a1, Arc a2
    WHERE a1.y = a2.x
    GROUP BY a1.x, a2.y;
    \end{verbatim}
    \begin{enumerate}[label=\alph*)]
        \item (1,2,4)
        \item (3,3,1)
        \item (2,4,6)
        \item (4,3,1)
    \end{enumerate}
    \textbf{Answer: d) (4,3,1)}
    \begin{itemize}
        \item \textit{Explanation:} This query performs a self-join on the `Arc` table where `a1.y = a2.x`, and then groups by `a1.x` and `a2.y` to count occurrences. Let's trace the join and grouping:
        \begin{itemize}
            \item The `WHERE a1.y = a2.x` condition looks for paths (x -> y -> z).
            \item Consider `a1.x = 4`. The tuples in `Arc` starting with 4 are (4,1), (4,1), (4,1), (4,2).
            \item If `a1.x = 4` and `a1.y = 1`, then `a2.x` must be 1. The tuples in `Arc` with `x = 1` are (1,2), (1,2). This would lead to (4,2) combinations. Since there are 3 instances of (4,1) in `a1` and 2 instances of (1,2) in `a2`, this contributes $3 \times 2 = 6$ to the count for (4,2).
            \item If `a1.x = 4` and `a1.y = 2`, then `a2.x` must be 2. The tuples in `Arc` with `x = 2` is (2,3). This leads to (4,3) combinations. Since there is 1 instance of (4,2) in `a1` and 1 instance of (2,3) in `a2`, this contributes $1 \times 1 = 1$ to the count for (4,3).
            \item Therefore, the tuple (4,3,1) is present in the result.
        \end{itemize}
    \end{itemize}

\item For any set of attributes X, the set X+ is a superkey.
    \begin{enumerate}[label=\alph*)]
        \item True
        \item False
    \end{enumerate}
    \textbf{Answer: b) False}
    \begin{itemize}
        \item \textit{Explanation:} X+ (the closure of X) represents all attributes that are functionally determined by X. While a key (and thus a superkey) must functionally determine all other attributes in a relation, the set X+ itself is not necessarily a superkey. A superkey is a set of attributes that *uniquely identifies* tuples. X+ tells you what attributes X can *determine*, not necessarily that X itself is a unique identifier. A superkey must contain a candidate key. X+ might not contain a candidate key, or X itself might not be minimal. For X to be a superkey, its closure X+ must include all attributes of the relation.
    \end{itemize}

\item Suppose relation R(A,B,C) has the tuples:
    Table:
    \begin{verbatim}
    A | B | C
    --|---|--
    0 | 1 | 2
    0 | 1 | 3
    4 | 5 | 6
    4 | 6 | 3
    \end{verbatim}
    Compute the bag union of the following three expressions, each of which is the bag projection of a grouping ($\gamma$) operation:

    1. $\pi$X($\gamma$A,B,MAX(C)$\rightarrow$X(R))
    2. $\pi$X($\gamma$B,SUM(C)$\rightarrow$X(R))
    3. $\pi$X($\gamma$A,MIN(B)$\rightarrow$X(R))

    Demonstrate that you have computed this bag correctly by identifying, from the list below, the correct count of occurrences for one of the elements.
    \begin{enumerate}[label=\alph*)]
        \item 3 appears exactly three times
        \item 6 appears exactly once
        \item 1 appears exactly twice
        \item 3 appears exactly twice
    \end{enumerate}
    \textbf{Answer: a) 3 appears exactly three times}
    \begin{itemize}
        \item \textit{Explanation:} Let's break down each expression:
        \begin{enumerate}
            \item $\pi$X($\gamma$A,B,MAX(C)$\rightarrow$X(R)):
                \begin{itemize}
                    \item Group by (A,B):
                        \item (0,1): MAX(C) is 3 (from (0,1,2) and (0,1,3))
                        \item (4,5): MAX(C) is 6 (from (4,5,6))
                        \item (4,6): MAX(C) is 3 (from (4,6,3))
                    \item Projection $\pi$X gives us: {3, 6, 3} (as a bag)
                \end{itemize}
            \item $\pi$X($\gamma$B,SUM(C)$\rightarrow$X(R)):
                \begin{itemize}
                    \item Group by B:
                        \item B=1: SUM(C) is 2+3=5 (from (0,1,2) and (0,1,3))
                        \item B=5: SUM(C) is 6 (from (4,5,6))
                        \item B=6: SUM(C) is 3 (from (4,6,3))
                    \item Projection $\pi$X gives us: {5, 6, 3} (as a bag)
                \end{itemize}
            \item $\pi$X($\gamma$A,MIN(B)$\rightarrow$X(R)):
                \begin{itemize}
                    \item Group by A:
                        \item A=0: MIN(B) is 1 (from (0,1,2) and (0,1,3))
                        \item A=4: MIN(B) is 5 (from (4,5,6) and (4,6,3))
                    \item Projection $\pi$X gives us: {1, 5} (as a bag)
                \end{itemize}
        \end{enumerate}
        Now, compute the bag union of {3, 6, 3}, {5, 6, 3}, and {1, 5}.
        Bag union means we combine all elements and keep their counts.
        Resulting Bag: {1, 3, 3, 3, 5, 5, 6, 6}
        \begin{itemize}
            \item 1 appears once.
            \item 3 appears three times.
            \item 5 appears twice.
            \item 6 appears twice.
        \end{itemize}
        Therefore, "3 appears exactly three times" is the correct statement.
    \end{itemize}

\item If the attribute K of a relation is a key, then no two tuples in the relation can have the same value of K.
    \begin{enumerate}[label=\alph*)]
        \item True
        \item False
    \end{enumerate}
    \textbf{Answer: a) True}
    \begin{itemize}
        \item \textit{Explanation:} This is the fundamental definition of a key in a relational database. A key (which can be a primary key or any other candidate key) is a set of attributes whose values uniquely identify each tuple (row) in a relation. If two tuples had the same value for K, then K would not be able to uniquely identify them, violating its definition as a key.
    \end{itemize}

\end{enumerate}

\end{document}
