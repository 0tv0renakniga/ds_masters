\documentclass{article}
\usepackage{amsmath}
\usepackage{amssymb}
\usepackage{amsthm}
\usepackage{enumitem}
\usepackage{geometry}
\geometry{a4paper, margin=1in}

% Package for code listings
\usepackage{listings}
\usepackage{xcolor}

% Configure SQL listing style
\lstdefinelanguage{SQL}{
  keywords={SELECT, FROM, WHERE, JOIN, ON, INNER, LEFT, RIGHT, FULL, OUTER, AS, DISTINCT, AND, OR, NOT, UNION, ALL},
  keywordstyle=\color{blue}\bfseries,
  identifierstyle=\color{black},
  stringstyle=\color{red},
  commentstyle=\color{green!60!black}\itshape,
  showstringspaces=false,
  tabsize=2,
  breaklines=true,
  columns=fullflexible,
  upquote=true,
  basicstyle=\ttfamily\small
}
\lstset{language=SQL}

\title{SQL Joins: Comprehensive Review}
\author{DSC 208R - Data Management for Analytics}
\date{}

\begin{document}

\maketitle

\section*{Introduction to SQL Joins}
SQL joins are used to combine rows from two or more tables based on a related column between them.

\subsection*{Schemas Used in Examples}
\begin{itemize}
    \item \textbf{Movie} (\underline{name}, year, genre) 
    \item \textbf{ActedIN} (\underline{actorname}, \underline{moviename}) 
    \item \textbf{Actor} (\underline{name}, age) 
\end{itemize}

\subsection*{Data for Examples}
\textbf{Movie Table:}\\
\begin{tabular}{|l|l|l|}
    \hline
    \textbf{Name} & \textbf{Year} & \textbf{Genre} \\
    \hline
    Apocalypse Now & 1979 & War \\
    The God Father & 1972 & Crime \\
    Planet Earth II & 2016 & Nature Documentary \\
    \hline
\end{tabular}\\

\textbf{ActedIN Table:}\\ 
\begin{tabular}{|l|l|}
    \hline
    \textbf{Actorname} & \textbf{Moviename} \\
    \hline
    Marlon Brando & Apocalypse Now \\
    Al Pacino & The God Father \\
    Marlon Brando & The God Father \\
    \hline
\end{tabular}\\

\textbf{Actor Table:}\\
\begin{tabular}{|l|l|}
    \hline
    \textbf{Name} & \textbf{Age} \\
    \hline
    Marlon Brando & 80 \\
    Al Pacino & 82 \\
    De Niro & 79 \\
    \hline
\end{tabular}\\

\section*{Types of SQL Joins}

\subsection*{Inner Join}
An \textbf{INNER JOIN} (or simply \texttt{JOIN}) returns only the rows that have matching values in both tables. Rows that do not have a match in the other table are excluded.

\textbf{Query:} What movies did each actor act in? 
\begin{lstlisting}
SELECT A.actorname, M.name
FROM ActedIN AS A JOIN Movie AS M
ON A.moviename = M.name;
\end{lstlisting}

\textbf{Result:}\\
\begin{tabular}{|l|l|}
    \hline
    \textbf{actorname} & \textbf{name} \\
    \hline
    Marlon Brando & Apocalypse Now \\
    Al Pacino & The God Father \\
    Marlon Brando & The God Father \\
    \hline
\end{tabular}

\textbf{Explanation:} The query joins `ActedIN` and `Movie` tables where `moviename` in `ActedIN` matches `name` in `Movie`. It then projects `actorname` and `name` (movie title) for the matched rows.

\subsection*{Left Outer Join}
A \textbf{LEFT OUTER JOIN} (or \texttt{LEFT JOIN}) returns all rows from the left table, and the matching rows from the right table. If there is no match in the right table, `NULL` values are returned for the right table's columns.

\textbf{Query:} Return all actors, and for each, the movies they acted in. 
\begin{lstlisting}
SELECT T1.name, T2.moviename
FROM Actor AS T1 LEFT OUTER JOIN ActedIN AS T2
ON T1.name = T2.actorname;
\end{lstlisting}

\textbf{Result:}\\ 
\begin{tabular}{|l|l|}
    \hline
    \textbf{name} & \textbf{moviename} \\
    \hline
    Marlon Brando & Apocalypse Now \\
    Marlon Brando & The God Father \\
    Al Pacino & The God Father \\
    De Niro & NULL \\
    \hline
\end{tabular}\\

\textbf{Explanation:} This query ensures all actors from the `Actor` table (left table) are included in the result. If an actor has no corresponding entries in the `ActedIN` table, `NULL` is displayed for `moviename`.

\subsection*{Right Outer Join}
A \textbf{RIGHT OUTER JOIN} (or \texttt{RIGHT JOIN}) returns all rows from the right table, and the matching rows from the left table. If there is no match in the left table, `NULL` values are returned for the left table's columns.

\textbf{Query:} Return all movies, and for each movie, the actors who acted in it. 
\begin{lstlisting}
SELECT T1.name, T2.actorname
FROM Movie AS T1 RIGHT OUTER JOIN ActedIN AS T2
ON T1.name = T2.moviename;
\end{lstlisting}

\textbf{Result:} \\
\begin{tabular}{|l|l|}
    \hline
    \textbf{name} & \textbf{actorname} \\
    \hline
    Apocalypse Now & Marlon Brando \\
    The God Father & Al Pacino \\
    The God Father & Marlon Brando \\
    NULL & NULL \\ % This specific NULL row for Planet Earth II is implied by the original PDF's explanation of RIGHT JOIN.
    % The source image only shows the ActedIN matches, but a true RIGHT JOIN with the given Movie data would imply this.
    \hline
\end{tabular}

\textbf{Explanation (clarified from original):} The query intends to show all entries from the `ActedIN` table (right table) and match them with `Movie` names. However, based on the provided table data in, `Planet Earth II` from the `Movie` table is not present in the `ActedIN.moviename` column. A correct `RIGHT JOIN` of `Movie` (T1) and `ActedIN` (T2) *on* `T1.name = T2.moviename` would include all `ActedIN` entries (as shown) and *also* any `Movie` entries that are not in `ActedIN`. If `ActedIN` has movie names not in `Movie`, they would appear with `NULL` for `Movie.name`. The provided example result focuses on the matched `ActedIN` entries, which might omit `NULL` rows for the Movie table if there are no `ActedIN` records for those movies. For instance, if 'Planet Earth II' had no actors, it would *not* appear in this specific result, as `ActedIN` is the right table. The example given in the PDF is actually incomplete for a full RIGHT JOIN illustration given the input data. A more typical result would show all rows from `ActedIN` (the right table), along with matching movie details from `Movie`. The sample output matches the `ActedIN` table exactly.

\subsection*{Full Outer Join}
A \textbf{FULL OUTER JOIN} (or \texttt{FULL JOIN}) returns all rows when there is a match in either the left (T1) or the right (T2) table. It includes:
\begin{itemize}
    \item All matching rows.
    \item All rows from the left table that have no match in the right table (with `NULL` for right table columns).
    \item All rows from the right table that have no match in the left table (with `NULL` for left table columns).
\end{itemize}

\textbf{Query:} 
\begin{lstlisting}
SELECT *
FROM Actor FULL OUTER JOIN ActedIN
ON Actor.name = ActedIN.actorname;
\end{lstlisting}

\textbf{Result:} 
\begin{tabular}{|l|l|l|l|}
    \hline
    \textbf{name} & \textbf{age} & \textbf{actorname} & \textbf{moviename} \\
    \hline
    Marlon Brando & 80 & Marlon Brando & Apocalypse Now \\
    Marlon Brando & 80 & Marlon Brando & The God Father \\
    Al Pacino & 82 & Al Pacino & The God Father \\
    De Niro & 79 & NULL & NULL \\
    NULL & NULL & NULL & NULL \\ % This NULL row implies an ActedIN entry without a matching Actor, which isn't in the provided data.
    \hline
\end{tabular}

\textbf{Explanation (clarified from original):} This query performs a full outer join between `Actor` and `ActedIN` tables on `Actor.name = ActedIN.actorname`. The result includes:
\begin{itemize}
    \item Matched rows (Marlon Brando, Al Pacino) with their age and acted movies.
    \item Rows from `Actor` that have no match in `ActedIN` (e.g., De Niro, who hasn't acted in listed movies), with `NULL` for `actorname` and `moviename` from `ActedIN`.
    \item *Theoretically*, rows from `ActedIN` that have no match in `Actor` (e.g., if there was an `ActedIN` entry for an actor not in the `Actor` table), with `NULL` for `Actor.name` and `Actor.age`. The example result shows a `NULL` row that might represent this, but based on the provided `Actor` and `ActedIN` data, all `actorname` values in `ActedIN` (`Marlon Brando`, `Al Pacino`) *do* exist in `Actor.name`. Thus, a `NULL, NULL, NULL, NULL` row isn't directly derivable from the given data, suggesting either an omitted example or an implicit case.

\end{itemize}

\section*{Union in SQL}
The \textbf{UNION} operator combines the result sets of two or more \texttt{SELECT} statements into a single result set.
\begin{itemize}
    \item Each \texttt{SELECT} statement within the `UNION` must have the same number of columns.
    \item The columns must have compatible data types.
    \item The columns must be in the same order.
    \item `UNION` by default removes duplicate rows. To include duplicates, use `UNION ALL`.
\end{itemize}

\textbf{Query:} Return all names from `Actor` and `Movie` tables 
\begin{lstlisting}
SELECT name FROM Actor
UNION
SELECT name FROM Movie;
\end{lstlisting}

\textbf{Result:}\\ 
\begin{tabular}{|l|}
    \hline
    \textbf{name} \\
    \hline
    Al Pacino \\
    Apocalypse Now \\
    De Niro \\
    Marlon Brando \\
    Planet Earth II \\
    The God Father \\
    \hline
\end{tabular}

\textbf{Explanation:} This query combines the `name` column from the `Actor` table and the `name` column from the `Movie` table. `UNION` ensures that only distinct names appear in the final result, regardless of which table they originated from.

\end{document}
