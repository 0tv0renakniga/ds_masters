\documentclass{article}
\usepackage{amsmath}
\usepackage{amssymb}
\usepackage{amsthm}
\usepackage{enumitem}
\usepackage{geometry}
\geometry{a4paper, margin=1in}

% Package for code listings
\usepackage{listings}
\usepackage{xcolor}

% Configure SQL listing style
\lstdefinelanguage{SQL}{
  keywords={SELECT, FROM, WHERE, INSERT, DELETE, UPDATE, CREATE, ALTER, DROP, TABLE, DISTINCT, AND, OR, NOT, JOIN, ON},
  keywordstyle=\color{blue}\bfseries,
  identifierstyle=\color{black},
  stringstyle=\color{red},
  commentstyle=\color{green!60!black}\itshape,
  showstringspaces=false,
  tabsize=2,
  breaklines=true,
  columns=fullflexible,
  upquote=true,
  basicstyle=\ttfamily\small
}
\lstset{language=SQL}

\title{SQL Basics: Comprehensive Review}
\author{DSC 208R: Data Management for Analytics}
\date{}

\begin{document}

\maketitle

\section*{Introduction to SQL}
SQL (Structured Query Language) is a powerful language used for managing and manipulating relational databases.

\subsection*{Components of SQL}
SQL is broadly categorized into:
\begin{itemize}
    \item \textbf{Data Definition Language (DDL)}: Used to create, alter, and delete tables and their attributes.
    \item \textbf{Data Manipulation Language (DML)}: Used to query one or more tables, and to insert, delete, or modify tuples (rows) in tables.
    \item \textbf{Triggers and Advanced Constraints}: Actions executed by the DBMS on updates, and used to specify complex integrity constraints.
\end{itemize}

\section*{Basic SQL Query Structure}
A basic SQL query follows this structure:
\begin{lstlisting}
SELECT [DISTINCT] <column expression list>
FROM <list of tables>
WHERE <predicate>
\end{lstlisting}
\begin{itemize}
    \item \texttt{SELECT}: Specifies columns to be retained in the results. The \texttt{DISTINCT} keyword (optional) ensures the resulting table does not have duplicate rows.
    \item \texttt{FROM}: Specifies the cross-product of tables from which data is retrieved.
    \item \texttt{WHERE}: Specifies selection conditions applied to the tables mentioned in the \texttt{FROM} clause.
\end{itemize}

\section*{SQL Operations Examples}

\subsection*{Schema Used in Examples}
\begin{itemize}
    \item \texttt{Movie} (\underline{name}, year, genre) 
    \item \texttt{ActedIN} (\underline{actorname}, \underline{moviename}) 
\end{itemize}

\subsection*{Projection in SQL}
\textbf{Query:}
\begin{lstlisting}
SELECT name, genre
FROM Movies
\end{lstlisting}
\textbf{Description:} Return movies names and their genres.

\textbf{Input Table (\texttt{Movies}):}
\begin{tabular}{|l|l|l|}
    \hline
    \textbf{Name} & \textbf{Year} & \textbf{Genre} \\
    \hline
    Apocalypse Now & 1989 & War \\
    The God Father & 1972 & Crime \\
    Planet Earth II & 2016 & Nature, Documentary \\
    \hline
\end{tabular}
\textbf{Resulting Table:}
\begin{tabular}{|l|l|}
    \hline
    \textbf{Name} & \textbf{Genre} \\
    \hline
    Apocalypse Now & War \\
    The God Father & Crime \\
    Planet Earth II & Nature, Documentary \\
    \hline
\end{tabular}


\subsection*{Selection in SQL}
\textbf{Query:}
\begin{lstlisting}
SELECT *
FROM Movies
WHERE year > 2000
\end{lstlisting}
\textbf{Description:} Return movies produced after 2000.

\textbf{Input Table (\texttt{Movies}):}
\begin{tabular}{|l|l|l|}
    \hline
    \textbf{Name} & \textbf{Year} & \textbf{Genre} \\
    \hline
    Apocalypse Now & 1989 & War \\
    The God Father & 1972 & Crime \\
    Planet Earth II & 2016 & Nature, Documentary \\
    \hline
\end{tabular}


\textbf{Resulting Table:}
\begin{tabular}{|l|l|l|}
    \hline
    \textbf{Name} & \textbf{Year} & \textbf{Genre} \\
    \hline
    Planet Earth II & 2016 & Nature, Documentary \\
    \hline
\end{tabular}


\subsection*{Selection and Projection in SQL}
\textbf{Query:}
\begin{lstlisting}
SELECT name
FROM Movies
WHERE year > 2000
\end{lstlisting}
\textbf{Description:} Return movie names produced after 2000.

\textbf{Input Table (\texttt{Movies}):}
\begin{tabular}{|l|l|l|}
    \hline
    \textbf{Name} & \textbf{Year} & \textbf{Genre} \\
    \hline
    Apocalypse Now & 1989 & War \\
    The God Father & 1972 & Crime \\
    Planet Earth II & 2016 & Nature, Documentary \\
    \hline
\end{tabular}


\textbf{Resulting Table:}
\begin{tabular}{|l|}
    \hline
    \textbf{Name} \\
    \hline
    Planet Earth II \\
    \hline
\end{tabular}


\subsection*{Joins in SQL}
\textbf{Query:}
\begin{lstlisting}
SELECT DISTINCT genre
FROM Movie, ActedIN
WHERE Movie.name = ActedIN.moviename
\end{lstlisting}

\textbf{Input Tables:}
\textbf{Movie Table:}
\begin{tabular}{|l|l|l|}
    \hline
    \textbf{Name} & \textbf{Year} & \textbf{Genre} \\
    \hline
    Apocalypse Now & 1979 & War \\
    The God Father & 1972 & Crime \\
    Planet Earth II & 2016 & Nature Documentary \\
    \hline
\end{tabular}


\textbf{ActedIN Table:}
\begin{tabular}{|l|l|}
    \hline
    \textbf{Actorname} & \textbf{Moviename} \\
    \hline
    Marlon Brando & Apocalypse Now \\
    Al Pacino & The God Father \\
    Marlon Brando & The God Father \\
    \hline
\end{tabular}


\textbf{What does this query return?} 

This query performs a join operation between the `Movie` and `ActedIN` tables based on the condition that `Movie.name` matches `ActedIN.moviename`. It then selects the distinct genres from the resulting joined table.

\textbf{Step-by-step breakdown:}
\begin{enumerate}
    \item \textbf{Cross-product (FROM Movie, ActedIN):} All combinations of rows from `Movie` and `ActedIN` tables are generated.
    \item \textbf{Selection (WHERE Movie.name = ActedIN.moviename):} Rows from the cross-product are filtered where the `name` from the `Movie` table matches the `moviename` from the `ActedIN` table.
        \begin{itemize}
            \item (Apocalypse Now, 1979, War) JOIN (Marlon Brando, Apocalypse Now)
            \item (The God Father, 1972, Crime) JOIN (Al Pacino, The God Father)
            \item (The God Father, 1972, Crime) JOIN (Marlon Brando, The God Father)
        \end{itemize}
    \item \textbf{Projection (SELECT DISTINCT genre):} From the filtered rows, only the `genre` column is selected, and duplicate genre values are removed.
\end{enumerate}

\textbf{Resulting Table (genres of movies that have actors listed in ActedIN):}
\begin{tabular}{|l|}
    \hline
    \textbf{Genre} \\
    \hline
    War \\
    Crime \\
    \hline
\end{tabular}

\end{document}
