\documentclass{article}
\usepackage{amsmath}
\usepackage{amssymb}
\usepackage{amsthm}
\usepackage{enumitem}
\usepackage{geometry}
\geometry{a4paper, margin=1in}

% Package for code listings
\usepackage{listings}
\usepackage{xcolor}

% Configure SQL listing style
\lstdefinelanguage{SQL}{
  keywords={CREATE, TABLE, INSERT, INTO, VALUES, ALTER, ADD, DROP, COLUMN, CONSTRAINT, PRIMARY KEY, FOREIGN KEY, REFERENCES, NOT NULL, UNIQUE, CHECK, DEFAULT, CASCADE, RESTRICT, SET NULL, SET DEFAULT, ON DELETE, ON UPDATE},
  keywordstyle=\color{blue}\bfseries,
  identifierstyle=\color{black},
  stringstyle=\color{red},
  commentstyle=\color{green!60!black}\itshape,
  showstringspaces=false,
  tabsize=2,
  breaklines=true,
  columns=fullflexible,
  upquote=true,
  basicstyle=\ttfamily\small
}
\lstset{language=SQL}

\title{SQL Constraints: Comprehensive Review}
\author{DSC 208R - Data Management for Analytics}
\date{}

\begin{document}

\maketitle

\section*{Introduction to SQL Constraints}
Constraints are rules enforced on data columns in a table. They are used to limit the type of data that can go into a table, ensuring the accuracy and reliability of the data. Constraints can be column level or table level.

\subsection*{Types of Constraints}
\begin{itemize}
    \item \textbf{Key Constraints}:
        \begin{itemize}
            \item `PRIMARY KEY` 
            \item `FOREIGN KEY` 
        \end{itemize}
    \item \textbf{Value Constraints}:
        \begin{itemize}
            \item `NOT NULL` 
            \item `UNIQUE` 
            \item `CHECK` 
            \item `DEFAULT` 
        \end{itemize}
\end{itemize}

\section*{Key Constraints}

\subsection*{`PRIMARY KEY`}
A `PRIMARY KEY` uniquely identifies each record in a table.
\begin{itemize}
    \item It must contain unique values.
    \item It cannot contain `NULL` values.
    \item A table can have only one primary key.
    \item It can consist of single or multiple columns.
\end{itemize}

\textbf{Example (Single Column Primary Key):} 
\begin{lstlisting}
CREATE TABLE Movie (
    name VARCHAR(50) PRIMARY KEY,
    year INT,
    genre VARCHAR(50)
);
\end{lstlisting}

\textbf{Example (Multi-column Primary Key):} 
\begin{lstlisting}
CREATE TABLE ActedIn (
    actorname VARCHAR(50),
    moviename VARCHAR(50),
    PRIMARY KEY (actorname, moviename)
);
\end{lstlisting}

\subsection*{`FOREIGN KEY`}
A `FOREIGN KEY` is a field (or collection of fields) in one table that refers to the `PRIMARY KEY` in another table.
\begin{itemize}
    \item It establishes a link between two tables.
    \item It ensures referential integrity, meaning that relationships between tables are consistent.
\end{itemize}

\textbf{Example:} 
\begin{lstlisting}
CREATE TABLE ActedIn (
    actorname VARCHAR(50),
    moviename VARCHAR(50),
    PRIMARY KEY (actorname, moviename),
    FOREIGN KEY (moviename) REFERENCES Movie(name)
);
\end{lstlisting}
\textbf{Explanation:} This constraint ensures that any `moviename` inserted into `ActedIn` must already exist as a `name` in the `Movie` table.

\subsubsection*{Referential Integrity Options for `FOREIGN KEY`} 
These options define what happens to foreign key rows when the referenced primary key row is deleted or updated.
\begin{itemize}
    \item \textbf{`ON DELETE` / `ON UPDATE`}: Specifies actions for delete or update operations on the referenced primary key.
        \begin{itemize}
            \item \textbf{`CASCADE`}: If a row in the parent table (with the primary key) is deleted or updated, the corresponding rows in the child table (with the foreign key) are also deleted or updated. 
            \item \textbf{`RESTRICT`}: Prevents the deletion or update of a parent row if there are dependent rows in the child table. This is often the default behavior. 
            \item \textbf{`SET NULL`}: Sets the foreign key column(s) in the child table to `NULL` when the parent row is deleted or updated. Requires the foreign key column to be nullable. 
            \item \textbf{`SET DEFAULT`}: Sets the foreign key column(s) in the child table to their default value when the parent row is deleted or updated. 
        \end{itemize}
\end{itemize}

\textbf{Example with `ON DELETE CASCADE`:} 
\begin{lstlisting}
CREATE TABLE ActedIn (
    actorname VARCHAR(50),
    moviename VARCHAR(50),
    PRIMARY KEY (actorname, moviename),
    FOREIGN KEY (moviename) REFERENCES Movie(name) ON DELETE CASCADE
);
\end{lstlisting}
\textbf{Explanation:} If a movie is deleted from the `Movie` table, all `ActedIn` records referencing that movie will automatically be deleted.

\section*{Value Constraints}

\subsection*{`NOT NULL`}
The `NOT NULL` constraint ensures that a column cannot have a `NULL` value.

\textbf{Example:} 
\begin{lstlisting}
CREATE TABLE Movie (
    name VARCHAR(50) PRIMARY KEY,
    year INT NOT NULL,
    genre VARCHAR(50)
);
\end{lstlisting}
\textbf{Explanation:} This ensures that every movie record must have a `year` value.

\subsection*{`UNIQUE`}
The `UNIQUE` constraint ensures that all values in a column (or a set of columns) are different.
\begin{itemize}
    \item It allows `NULL` values, but only once (behavior can vary slightly by DBMS). 
    \item A table can have multiple `UNIQUE` constraints. 
\end{itemize}

\textbf{Example:} 
\begin{lstlisting}
CREATE TABLE Person (
    id INT PRIMARY KEY,
    email VARCHAR(100) UNIQUE,
    name VARCHAR(50)
);
\end{lstlisting}
\textbf{Explanation:} This ensures that no two persons can have the same `email`.

\subsection*{`CHECK`}
The `CHECK` constraint is used to limit the value range that can be placed in a column.

\textbf{Example:} 
\begin{lstlisting}
CREATE TABLE Actor (
    name VARCHAR(50) PRIMARY KEY,
    age INT CHECK (age > 0 AND age < 120)
);
\end{lstlisting}
\textbf{Explanation:} This ensures that the `age` of an actor must be between 1 and 119, inclusive.

\subsection*{`DEFAULT`}
The `DEFAULT` constraint is used to set a default value for a column when no value is specified during an `INSERT` operation.

\textbf{Example:} 
\begin{lstlisting}
CREATE TABLE Movie (
    name VARCHAR(50) PRIMARY KEY,
    year INT NOT NULL,
    genre VARCHAR(50) DEFAULT 'Unspecified'
);
\end{lstlisting}
\textbf{Explanation:} If a `genre` is not provided when inserting a new movie, it will automatically be set to 'Unspecified'.

\section*{Adding and Dropping Constraints with `ALTER TABLE`}

\subsection*{Adding a `PRIMARY KEY`} 
\begin{lstlisting}
ALTER TABLE Movie
ADD PRIMARY KEY (name);
\end{lstlisting}

\subsection*{Adding a `FOREIGN KEY`}
\begin{lstlisting}
ALTER TABLE ActedIn
ADD FOREIGN KEY (moviename) REFERENCES Movie(name);
\end{lstlisting}

\subsection*{Adding a `CHECK` Constraint}
\begin{lstlisting}
ALTER TABLE Actor
ADD CHECK (age > 0 AND age < 120);
\end{lstlisting}

\subsection*{Dropping a `PRIMARY KEY`}
\begin{lstlisting}
ALTER TABLE Movie
DROP PRIMARY KEY;
\end{lstlisting}

\subsection*{Dropping a `FOREIGN KEY`}
\begin{lstlisting}
ALTER TABLE ActedIn
DROP FOREIGN KEY moviename; -- Actual constraint name might be needed for some DBMS
\end{lstlisting}
\textbf{Note:} For some DBMS, you might need to specify the system-generated or user-defined name of the foreign key constraint instead of the column name.

\subsection*{Dropping a `CHECK` Constraint}
\begin{lstlisting}
ALTER TABLE Actor
DROP CHECK age_check; -- Assuming 'age_check' is the constraint name
\end{lstlisting}
\textbf{Note:} You typically need the specific name of the `CHECK` constraint to drop it.

\end{document}
