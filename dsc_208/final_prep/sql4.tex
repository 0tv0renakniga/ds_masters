\documentclass{article}
\usepackage{amsmath}
\usepackage{amssymb}
\usepackage{amsthm}
\usepackage{enumitem}
\usepackage{geometry}
\geometry{a4paper, margin=1in}

% Package for code listings
\usepackage{listings}
\usepackage{xcolor}

% Configure SQL listing style
\lstdefinelanguage{SQL}{
  keywords={SELECT, FROM, WHERE, IN, NOT IN, EXISTS, NOT EXISTS, UNIQUE, ALL, ANY, SOME, DISTINCT, AS, AND, OR, NOT, MAX, MIN, AVG, SUM, COUNT},
  keywordstyle=\color{blue}\bfseries,
  identifierstyle=\color{black},
  stringstyle=\color{red},
  commentstyle=\color{green!60!black}\itshape,
  showstringspaces=false,
  tabsize=2,
  breaklines=true,
  columns=fullflexible,
  upquote=true,
  basicstyle=\ttfamily\small
}
\lstset{language=SQL}

\title{SQL Nested Queries: Comprehensive Review}
\author{DSC 208R - Data Management for Analytics}
\date{June 2025}

\begin{document}

\maketitle

\section*{Introduction to Nested Queries}
A nested query (or subquery) is a query within another SQL query. Subqueries can be used in the `SELECT`, `FROM`, or `WHERE` clauses.

\subsection*{Schemas Used in Examples}
\begin{itemize}
    \item \textbf{Movie} (\underline{name}, year, genre) 
    \item \textbf{ActedIN} (\underline{actorname}, \underline{moviename}) 
    \item \textbf{Actor} (\underline{name}, age) 
\end{itemize}

\subsection*{Data for Examples}
\textbf{Movie Table:} 
\begin{tabular}{|l|l|l|}
    \hline
    \textbf{Name} & \textbf{Year} & \textbf{Genre} \\
    \hline
    Apocalypse Now & 1979 & War \\
    The God Father & 1972 & Crime \\
    Planet Earth II & 2016 & Nature Documentary \\
    \hline
\end{tabular}

\textbf{ActedIN Table:} 
\begin{tabular}{|l|l|}
    \hline
    \textbf{Actorname} & \textbf{Moviename} \\
    \hline
    Marlon Brando & Apocalypse Now \\
    Al Pacino & The God Father \\
    Marlon Brando & The God Father \\
    \hline
\end{tabular}

\textbf{Actor Table:} 
\begin{tabular}{|l|l|}
    \hline
    \textbf{Name} & \textbf{Age} \\
    \hline
    Marlon Brando & 80 \\
    Al Pacino & 82 \\
    De Niro & 79 \\
    \hline
\end{tabular}

\section*{Types of Nested Queries}

\subsection*{`IN` / `NOT IN` }
These operators check if a value is present (or not present) within the result set of a subquery.

\textbf{Query:} Find all actors who acted in at least one movie. 
\begin{lstlisting}
SELECT name
FROM Actor
WHERE name IN (SELECT actorname FROM ActedIN);
\end{lstlisting}
\textbf{Result:} 
\begin{tabular}{|l|}
    \hline
    \textbf{name} \\
    \hline
    Marlon Brando \\
    Al Pacino \\
    \hline
\end{tabular}
\textbf{Explanation:} The subquery `SELECT actorname FROM ActedIN` returns a list of actor names who have acted in movies. The outer query then selects names from the `Actor` table that are present in this list.

\subsection*{`EXISTS` / `NOT EXISTS` }
These operators check for the existence of rows returned by a subquery. They return true if the subquery returns any rows, and false otherwise. The subquery typically refers to attributes from the outer query (correlated subquery).

\textbf{Query:} Find all actors who acted in at least one movie. (Equivalent to `IN` example) 
\begin{lstlisting}
SELECT name
FROM Actor AS T1
WHERE EXISTS (SELECT * FROM ActedIN AS T2 WHERE T1.name = T2.actorname);
\end{lstlisting}
\textbf{Result:} 
\begin{tabular}{|l|}
    \hline
    \textbf{name} \\
    \hline
    Marlon Brando \\
    Al Pacino \\
    \hline
\end{tabular}
\textbf{Explanation:} For each actor in `Actor` (aliased as `T1`), the subquery checks if there exists any record in `ActedIN` (aliased as `T2`) where the actor's name matches. If such a record exists, the outer query includes the actor's name.

\subsection*{`UNIQUE` }
The `UNIQUE` operator returns true if a subquery produces no duplicate rows.

\textbf{Query:} Find movies that have only one actor listed. 
\begin{lstlisting}
SELECT name
FROM Movie
WHERE UNIQUE (SELECT moviename FROM ActedIN WHERE moviename = Movie.name);
\end{lstlisting}
\textbf{Explanation:} This query would look for movies where the subquery `SELECT moviename FROM ActedIN WHERE moviename = Movie.name` (for each movie) returns only one instance of that `moviename`, implying only one actor acted in it. Based on the provided `ActedIN` table, 'Apocalypse Now' has one entry, and 'The God Father' has two (Marlon Brando, Al Pacino).

\subsection*{`ALL` }
The `ALL` operator is used with comparison operators (`>`, `<`, `=`, etc.) and returns true if the comparison is true for all values in the result set of the subquery.

\textbf{Query:} Find actors who are older than all actors who acted in 'The God Father'. 
\begin{lstlisting}
SELECT name
FROM Actor
WHERE age > ALL (SELECT T1.age FROM Actor AS T1, ActedIN AS T2 WHERE T1.name = T2.actorname AND T2.moviename = 'The God Father');
\end{lstlisting}
\textbf{Explanation:} The subquery finds the ages of all actors in 'The God Father' (Al Pacino: 82, Marlon Brando: 80). The outer query then selects actors whose age is greater than *all* of these ages (i.e., greater than 82). In this dataset, no actor is older than 82.

\subsection*{`ANY` / `SOME` }
The `ANY` (or `SOME`) operator is used with comparison operators (`>`, `<`, `=`, etc.) and returns true if the comparison is true for at least one value in the result set of the subquery.

\textbf{Query:} Find actors who are older than at least one actor who acted in 'The God Father'. 
\begin{lstlisting}
SELECT name
FROM Actor
WHERE age > ANY (SELECT T1.age FROM Actor AS T1, ActedIN AS T2 WHERE T1.name = T2.actorname AND T2.moviename = 'The God Father');
\end{lstlisting}
\textbf{Explanation:} The subquery returns ages 82 (Al Pacino) and 80 (Marlon Brando). The outer query selects actors whose age is greater than *any* of these ages.
\begin{itemize}
    \item De Niro (79) is not `> ANY` (82, 80).
    \item Marlon Brando (80) is not `> ANY` (82, 80).
    \item Al Pacino (82) is not `> ANY` (82, 80).
\end{itemize}
In this specific dataset, no actor is strictly `> ANY` of (82, 80) if they are themselves one of those actors. For example, Al Pacino (82) is not greater than 82, and 80 is not greater than 82. If there was an actor aged 85, they would be included.

\section*{Correlated vs. Non-Correlated Subqueries}
\begin{itemize}
    \item \textbf{Non-correlated subquery:} A subquery that can be executed independently of the outer query. It typically runs once and provides its result set to the outer query. Examples include the `IN` clause used previously.
    \item \textbf{Correlated subquery:} A subquery that depends on the outer query for its values and therefore executes once for each row processed by the outer query. `EXISTS`, `NOT EXISTS`, `UNIQUE`, `ALL`, and `ANY` typically involve correlated subqueries.
\end{itemize}

\end{document}
