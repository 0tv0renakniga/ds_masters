\documentclass{article}
\usepackage{amsmath}
\usepackage{amssymb}
\usepackage{amsthm}
\usepackage{enumitem}
\usepackage{geometry}
\geometry{a4paper, margin=1in}

% Relational Algebra symbols
\newcommand{\sel}{\sigma} % Selection
\newcommand{\proj}{\pi} % Projection
\newcommand{\ren}{\rho} % Rename
\newcommand{\join}{\bowtie} % Natural Join
\newcommand{\ljoin}{\Join} % Left Outer Join (custom)
\newcommand{\rjoin}{\rightouterjoin} % Right Outer Join (custom)
\newcommand{\fjoin}{\fullouterjoin} % Full Outer Join (custom)
\newcommand{\cross}{\times} % Cartesian Product
\newcommand{\union}{\cup} % Union
\newcommand{\intersect}{\cap} % Intersection
\newcommand{\diff}{\setminus} % Set Difference (Minus)
\newcommand{\dupelim}{\delta} % Duplicate Elimination
\newcommand{\group}{\gamma} % Grouping
\newcommand{\sort}{\tau} % Sorting

% Package for code listings
\usepackage{listings}
\usepackage{xcolor}

% Configure SQL listing style
\lstdefinelanguage{SQL}{
  keywords={SELECT, FROM, WHERE, JOIN, ON, INNER, LEFT, RIGHT, FULL, OUTER, AS, DISTINCT, AND, NOT, EXISTS, EXCEPT, UNION},
  keywordstyle=\color{blue}\bfseries,
  identifierstyle=\color{black},
  stringstyle=\color{red},
  commentstyle=\color{green!60!black}\itshape,
  showstringspaces=false,
  tabsize=2,
  breaklines=true,
  columns=fullflexible,
  upquote=true,
  basicstyle=\ttfamily\small
}
\lstset{language=SQL}

\title{Formal Query Languages Part 2: Comprehensive Review}
\author{DSC 208R - Data Management for Analytics}
\date{}

\begin{document}

\maketitle

\section*{Combining Relational Algebra Operators}
Relational algebra operators can be combined to form complex queries, similar to how expressions are built in programming languages. Precedence rules dictate the order of evaluation. 

\subsection*{Operator Precedence (Highest to Lowest)} 
\begin{enumerate}
    \item Unary operators: $\ren$, $\proj$, $\sel$ 
    \item Cartesian product and Joins: $\times$, $\join$ 
    \item Intersection: $\intersect$ 
    \item Union and Set Difference: $\union$, $\setminus$ 
\end{enumerate}

\subsection*{Combining Operators Example}

Consider the Movie table: \\

\begin{tabular}{|l|l|l|l|l|}
    \hline
    \textbf{Name} & \textbf{Year} & \textbf{Genre} & \textbf{Budget} & \textbf{Revenue} \\
    \hline
    Pirates of the Caribbean & 2007 & Action & \$300 m & \$900 m \\
    The Lion King & 2019 & Animation & \$260 m & \$1.65 b \\
    The Dark Knight & 2008 & Action & \$185 m & \$1 b \\
    \hline
\end{tabular}\\

\textbf{Projection example:} $\proj_{\text{name,rate}}(\text{Movie})$ \\

This projects the `name` and `rate` columns. \\

\begin{tabular}{|l|l|}
    \hline
    \textbf{Name} & \textbf{Rate} \\
    \hline
    Pirates of the Caribbean & 7.1 \\
    The Lion King & 6.5 \\
    The Dark Knight & 9.5 \\
    \hline
\end{tabular}\\


\textbf{Selection example:} $\sel_{\text{budget}>\$200m}(\text{Movie})$ \\

This selects movies with a budget greater than \$200 million.\\

\begin{tabular}{|l|l|l|l|l|}
    \hline
    \textbf{Name} & \textbf{Year} & \textbf{Genre} & \textbf{Budget} & \textbf{Revenue} \\
    \hline
    Pirates of the Caribbean & 2007 & Action & \$300 m & \$900 m \\
    The Lion King & 2019 & Animation & \$260 m & \$1.65 b \\
    \hline
\end{tabular}\\


\textbf{Combined example:} $\proj_{\text{name,rate}}(\sel_{\text{budget}>\$200m}(\text{Movie}))$ \\

This first selects movies by budget, then projects name and rate.\\

\begin{tabular}{|l|l|}
    \hline
    \textbf{Name} & \textbf{Rate} \\
    \hline
    Pirates of the Caribbean & 7.1 \\
    The Lion King & 6.5 \\
    \hline
\end{tabular}\\


\section*{Outer Joins} 
Outer joins keep all information from both operands, padding with `NULL` values for tuples that have no counterpart in the other relation. 

\subsection*{Three Variants of Outer Join} 
\begin{itemize}
    \item \textbf{Left Outer Join}: Keeps all rows from the left table. 
    \item \textbf{Right Outer Join}: Keeps all rows from the right table. 
    \item \textbf{Full Outer Join}: Keeps all rows from both tables. 
\end{itemize}

\textbf{Notation for Outer Joins:} 
Relational Algebra: $R1 \ljoin_{R1.A1=R2.A2} R2$ (Left Join)
SQL: `SELECT * FROM R1 LEFT JOIN R2 ON R1.A1 = R2.A2`
Pandas: $df1.merge(df2, left_on='A1', right_on='A2', how='left')$

\textbf{Outer Join Example:} 

\textbf{Movie Table (M):} \\

\begin{tabular}{|l|l|l|}
    \hline
    \textbf{Name} & \textbf{Year} & \textbf{Genre} \\
    \hline
    Apocalypse Now & 1979 & War \\
    The God Father & 1972 & Crime \\
    Planet Earth II & 2016 & Nature Documentary \\
    \hline
\end{tabular}\\

\textbf{ActedIN Table (A):} \\

\begin{tabular}{|l|l|}
    \hline
    \textbf{Actorname} & \textbf{Moviename} \\
    \hline
    Marlon Brando & Apocalypse now \\
    Al Pacino & The God Father \\
    Marlon Brando & The God Father \\
    \hline
\end{tabular}\\


\textbf{Result of `Movie` Left Outer Join `ActedIN` on `Movie.name = ActedIN.moviename`:} \\

\begin{tabular}{|l|l|l|l|l|}
    \hline
    \textbf{Name} & \textbf{Year} & \textbf{Genre} & \textbf{Actorname} & \textbf{Moviename} \\
    \hline
    Apocalypse now & 1979 & War & Marlon Brando & Apocalypse Now \\
    The God Father & 1972 & Crime & Al Pacino & The God Father \\
    Planet Earth II & 2016 & Nature Documentary & NULL & NULL \\
    \hline
\end{tabular}\\


\section*{Relational Algebra for Bags} 
SQL uses bag semantics, meaning duplicate tuples are preserved. Relational engines often work on bags for efficiency. \\


\subsection*{Extended Relational Algebra Operators on Bags} 
\begin{itemize}
    \item \textbf{Selection ($\sel_\theta(R)$)}: Preserves the number of occurrences of tuples. 
    \item \textbf{Projection ($\proj_A(R)$)}: No duplicate elimination (unlike set-based projection). 
    \item \textbf{Cartesian product, Join}: No duplicate elimination. 
    \item \textbf{Duplicate elimination ($\dupelim(R)$)}: `R1 := $\dupelim(R2)$` - R1 consists of one copy of each tuple that appears in R2 one or more times. 
    \item \textbf{Grouping ($\group$)}: A list of elements that are either individual (grouping) attributes or `AGG(A)` (aggregation operator on attribute A). 
    \item \textbf{Sorting ($\sort$)}: `R1 := $\sort_L(R2)$` - R1 consists of tuples of R2 sorted on the attributes in list L. 
\end{itemize}

\subsection*{Grouping / Aggregation Example} 

Input `Movie` table (with `Revenue`):\\

\begin{tabular}{|l|l|l|l|}
    \hline
    \textbf{Name} & \textbf{Year} & \textbf{Genre} & \textbf{Revenue} \\
    \hline
    Pirates of the Caribbean & 2007 & Action & \$900 m \\
    The Lion King & 2019 & Animation & \$1.65 b \\
    The Dark Knight & 2008 & Action & \$1 b \\
    Toy Story 3 & 2010 & Animation & \$1 b \\
    American Sniper & 2013 & Action & \$350 m \\
    \hline
\end{tabular}\\


Grouping by `Genre` and calculating `SUM(Revenue)` as `TotalRevenue`, and `COUNT(*)` as `MovieNum`:\\

$\group_{\text{genre, SUM(Revenue) \(\rightarrow\) TotalRevenue, COUNT(*) \(\rightarrow\) MovieNum}}(\text{Movie})$ \\

Result:\\

\begin{tabular}{|l|l|l|}
    \hline
    \textbf{Genre} & \textbf{TotalRevenue} & \textbf{MovieNum} \\
    \hline
    Animation & \$2.65 b & 2 \\
    Action & \$2.25 b & 3 \\
    \hline
\end{tabular}\\

\section*{Representing Relational Algebra Queries as Trees} 

Complex RA queries can be represented as expression trees, which are useful for query optimization.\\

\subsection*{Example 1: Select-Project-Join Query} 

SQL Query: \\

\begin{lstlisting}[language=SQL]
SELECT genre
FROM Movie, ActedIN
WHERE Movie.name = ActedIN.moviename AND
      ActedIN.actorname = 'Marlon Brando';
\end{lstlisting}
Relational Algebra Equivalent:
$\proj_{\text{genre}}(\text{Movie} \join (\sel_{\text{actorname='Marlon Brando'}}(\text{ActedIN})))$ 

\subsection*{Example 2: Grouping and Aggregation Query} 

SQL Query: \\

\begin{lstlisting}[language=SQL]
SELECT genre, SUM(revenue) AS TR, COUNT(*) AS C
FROM Movie
WHERE year > 2008
GROUP BY genre
HAVING SUM(revenue) > $1B;
\end{lstlisting}

Relational Algebra Representation (simplified tree notation): \\

The query involves selection, grouping with aggregation, and then a filtering (having) operation on the grouped results, followed by projection. \\

\section*{Typical Query Plan} 
\begin{itemize}
    \item For `SELECT-PROJECT-JOIN` queries, the plan generally involves selections, joins, and then projection. 
    \item For queries with `GROUP BY` and `HAVING`, the order is typically: `FROM` (identifying tables), `WHERE` (filtering rows), `GROUP BY` (grouping), `HAVING` (filtering groups), and finally `SELECT` (projecting fields and aggregates). 
\end{itemize}

\section*{Subqueries and Relational Algebra} 
SQL subqueries can often be de-correlated and expressed using relational algebra set operations.

\subsection*{Example: `EXISTS` and `NOT EXISTS`} 
SQL Query: Find actors who acted in an 'Action' movie but NOT a 'Drama' movie.
\begin{lstlisting}[language=SQL]
SELECT a.actorname
FROM ActedIn a
WHERE EXISTS (SELECT a.actorname FROM Movie m WHERE a.moviename = m.name AND m.genre = 'Action')
  AND NOT EXISTS (SELECT a.actorname FROM Movie m WHERE a.moviename = m.name AND m.genre = 'Drama');
\end{lstlisting}
This query uses correlated subqueries. 

\subsection*{Example: `EXCEPT` (Set Difference)} 
The previous query can be rewritten using `EXCEPT` (set difference): 
\begin{lstlisting}[language=SQL]
SELECT a1.actorname
FROM Movie m1, ActedIN a1
WHERE m1.name = a1.moviename AND m1.genre = 'Action'
EXCEPT
SELECT a2.actorname
FROM Movie m2, ActedIN a2
WHERE m2.name = a2.moviename AND m2.genre = 'Drama';
\end{lstlisting}
Relational Algebra Equivalent: 
$\proj_{\text{actorname}}(\text{Movie} \join \sel_{\text{genre='Action'}}(\text{ActedIN})) \setminus \proj_{\text{actorname}}(\text{Movie} \join \sel_{\text{genre='Drama'}}(\text{ActedIN}))$

\section*{Summary} 
\begin{itemize}
    \item SQL is a declarative language (describes *what* to get). 
    \item Relational Algebra (RA) is a procedural language (describes *how* to get it). 
    \item SQL and Relational Algebra express the same class of queries (they are equivalent in expressive power for core operations). 
\end{itemize}

\end{document}
