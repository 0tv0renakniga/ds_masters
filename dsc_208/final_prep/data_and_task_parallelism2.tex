\documentclass{article}
\usepackage{amsmath}
\usepackage{amssymb}
\usepackage{amsthm}
\usepackage{enumitem}
\usepackage{geometry}
\geometry{a4paper, margin=1in}

\title{Basics of Parallelism: Comprehensive Review}
\author{Based on DSC 208R – Data Management for Analytics by Arun Kumar, UC San Diego}
\date{June 2025}

\begin{document}

\maketitle

\section*{Parallel Data Processing: Basic Idea}
The fundamental concept behind parallel data processing is to split up a large workload across multiple processors, and often across multiple machines or workers. This is essentially a "Divide and Conquer" strategy. The central issue addressed by parallelism is that a given workload takes too long for a single processor to complete efficiently.

\subsection*{Threads}
In parallel data processing, "threads" are a common concept, generalizing the process abstraction of an operating system (OS).
\begin{itemize}
    \item A program or process can spawn many threads.
    \item Each thread runs its part of the program's computations simultaneously.
    \item All threads within a process share the same address space (and thus, the same data).
    \item In multi-core CPUs, typically one thread uses one core. "Hyper-threading" can virtualize a core to run two threads simultaneously.
\end{itemize}

\section*{Parallelism Paradigms}
Several paradigms describe how parallel computations are structured.

\subsection*{1. SIMD (Single Instruction, Multiple Data)}
\begin{itemize}
    \item A single instruction operates on multiple data items concurrently.
    \item Examples include vector processing units in CPUs (e.g., Intel AVX), GPUs, and TPUs.
    \item Common in modern CPUs and specialized accelerators for tasks like matrix multiplication and image processing.
\end{itemize}

\subsection*{2. MIMD (Multiple Instruction, Multiple Data)}
\begin{itemize}
    \item Multiple processors execute different instructions on different data streams simultaneously.
    \item Most modern multi-core CPUs and distributed clusters fall under this paradigm.
    \item Sub-types include Shared-Memory MIMD and Distributed-Memory MIMD.
\end{itemize}

\subsection*{3. SPMD (Single Program, Multiple Data)}
\begin{itemize}
    \item A single program is executed by multiple processors, each operating on a different portion of the data.
    \item Each processor executes the same program, but can follow different control paths based on its local data.
    \item This is very common in distributed data systems like Hadoop MapReduce and Apache Spark.
\end{itemize}

\subsection*{4. MPP (Massively Parallel Processing)}
\begin{itemize}
    \item A system with many independent processors, each with its own memory and operating system, working on a single application.
    \item Often associated with shared-nothing architectures in databases (e.g., Teradata, Greenplum, CitusDB).
    \item The terms "MPP" and "Shared-Nothing Parallelism" are often used interchangeably.
\end{itemize}

\section*{Amdahl's Law}
Amdahl's Law is a formula that gives the theoretical maximum speedup for latency of execution of a fixed workload as parallel resources are added. It highlights the diminishing returns of parallelism.

\begin{itemize}
    \item Let $S$ be the fraction of the program that is inherently sequential (cannot be parallelized).
    \item Then $(1-S)$ is the fraction that can be parallelized.
    \item The maximum speedup ($P$) achievable with $N$ processors is given by:
    \[
    P = \frac{1}{S + \frac{1-S}{N}}
    \]
    \item As $N \to \infty$, the maximum speedup approaches $\frac{1}{S}$.
\end{itemize}
\textbf{Implication:} Even a small sequential portion significantly limits the overall speedup, making it crucial to minimize sequential parts for highly parallel systems.

\section*{Dataflow Graph / Task Graph}
Parallel computations are often represented as Directed Acyclic Graphs (DAGs).

\subsection*{Dataflow Graph (aka Logical Query Plan / Neural Computational Graph)}
\begin{itemize}
    \item A DAG where vertices represent operators (e.g., from extended relational algebra or linear algebra/deep learning tool's tensor algebra).
    \item Edges represent the flow of data between these operators.
    \item In database systems, it's known as a Logical Query Plan.
    \item In machine learning systems, it's known as a Neural Computational Graph (e.g., for $ReLU(WX+b)$).
\end{itemize}

\subsection*{Task Graph}
\begin{itemize}
    \item A DAG where vertices represent full tasks or processes (more coarse-grained than operator-level dataflows).
    \item Edges represent dependencies between tasks.
    \item Dask, for example, conflates the concepts of Dataflow and Task graphs, where an operation on a Dask DataFrame becomes its own separate process/program.
    \item Cycles in task graphs are possible for iterative computations, making them "cyclic graphs" or "recurrent graphs."
\end{itemize}

\end{document}
