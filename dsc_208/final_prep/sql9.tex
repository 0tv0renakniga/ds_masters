\documentclass{article}
\usepackage{amsmath}
\usepackage{amssymb}
\usepackage{amsthm}
\usepackage{enumitem}
\usepackage{geometry}
\geometry{a4paper, margin=1in}

% Package for code listings
\usepackage{listings}
\usepackage{xcolor}

% Configure SQL listing style
\lstdefinelanguage{SQL}{
  keywords={SELECT, FROM, WHERE, CREATE, INDEX, ON, USING, btree, hash, INSERT, UPDATE, DELETE, LIKE},
  keywordstyle=\color{blue}\bfseries,
  identifierstyle=\color{black},
  stringstyle=\color{red},
  commentstyle=\color{green!60!black}\itshape,
  showstringspaces=false,
  tabsize=2,
  breaklines=true,
  columns=fullflexible,
  upquote=true,
  basicstyle=\ttfamily\small
}
\lstset{language=SQL}

\title{Database Indexes: Comprehensive Review}
\author{DSC 208R - Data Management for Analytics}
\date{}

\begin{document}

\maketitle

\section*{Motivation for Indexing}
Consider typical SQL queries that retrieve records based on specific conditions, such as `SELECT * FROM Movies WHERE Year=2017` or `SELECT * FROM Movies WHERE Year>=2000 AND Year<2010`.
\begin{itemize}
    \item The naive option to obtain matching records is a full filescan of the table, applying the predicate to each tuple.
    \item This approach has an $O(N)$ cost for both I/O (input/output) and CPU, where $N$ is the number of tuples in the table.
\end{itemize}

\section*{Overview of Indexes}
An index is an auxiliary data structure in a database that helps locate and retrieve records much faster than an $O(N)$ filescan.
\begin{itemize}
    \item \textbf{IndexKey (SearchKey)}: Refers to the attribute(s) on which a file is indexed. It can be any permutation of any subset of a relation's attributes and does not necessarily have to be a primary or candidate key.
    \item RDBMSs support indexes designed and optimized for search, insert, and delete operations in SQL.
    \item In practice, queries can become millions of times faster when suitable indexes are available on a table.
\end{itemize}

\subsection*{Comparison: Indexed Access vs. Filescan}
\begin{itemize}
    \item **Filescan**: Reads every page of the table. This is necessary for queries like `SELECT COUNT(*) FROM Movies` or `SELECT * FROM Movies WHERE Name LIKE '%Lion%'`. It is suitable for analytical queries that process large portions of data.
    \item **Indexed Access**: Reads only the relevant pages of the table via the index. This is beneficial for queries that involve `WHERE` conditions on the indexed attribute(s).
\end{itemize}

\subsection*{Types of Indexes}
\subsubsection*{1. B+Tree Indexes}
\begin{itemize}
    \item B+Trees are multi-level indexes, forming a balanced tree structure.
    \item They are the most common type of index in RDBMSs due to their efficiency in handling range queries.
    \item Each node in a B+Tree is typically a disk block.
    \item All data records are stored at the leaf level, which are linked lists.
    \item B+Trees support efficient retrieval for equality searches, range searches, and ordered retrieval.
\end{itemize}

\textbf{Example SQL for B+Tree Index Creation:}
\begin{lstlisting}[language=SQL]
CREATE INDEX MIn1 ON Movies USING btree (Year);
\end{lstlisting}
This index is useful for `WHERE year = 2017` and `WHERE year >= 2000 AND year < 2010`.

\textbf{Composite B+Tree Indexes:}
If a B+Tree index has a composite `IndexKey` (multiple attributes), the order of attributes matters; it is a sequence.
\begin{lstlisting}[language=SQL]
CREATE INDEX MIn1 ON Movies USING btree (Name, Year);
CREATE INDEX MIn2 ON Movies USING btree (Year, Name);
\end{lstlisting}
\begin{itemize}
    \item An index on `(Year, Name)` (`MIn2`) is useful for queries like `SELECT * FROM Movies WHERE Year >= 2015` because of the prefix match on `Year`.
    \item An index on `(Name, Year)` (`MIn1`) is generally useless for a query like `SELECT * FROM Movies WHERE Year >= 2015` because the `Year` attribute is not a prefix of the `IndexKey`.
\end{itemize}

\subsubsection*{2. Hash Indexes}
\begin{itemize}
    \item Hash indexes are single-level indexes that use a hash function to map `IndexKey` values to data record locations.
    \item They are generally efficient for equality searches (`=`).
    \item They are NOT efficient for range queries (`>`, `<`, `BETWEEN`) or `LIKE` pattern matching with wildcards at the beginning.
\end{itemize}

\textbf{Example SQL for Hash Index Creation:}
\begin{lstlisting}[language=SQL]
CREATE INDEX MIn2 ON Movies USING hash (Name);
\end{lstlisting}
This index is useful for `WHERE name = 'The Lion King'` but not `WHERE name LIKE '%Lion%'`.

\textbf{Composite Hash Indexes:}
If a hash index has a composite `IndexKey`, the order among attributes does not matter; it is a set.

\section*{When to Use Indexes}
Indexes are beneficial for:
\begin{itemize}
    \item Attributes frequently used in `WHERE` clause predicates.
    \item Attributes involved in join conditions.
    \item Attributes used for `GROUP BY` or `ORDER BY` clauses.
    \item Foreign key columns, as they are often used in join conditions.
    \item When the table is large and the queries retrieve a small percentage of its records.
\end{itemize}

\subsection*{Indexing for DML Operations}
While indexes speed up reads, they slow down writes (INSERT, UPDATE, DELETE).
\begin{itemize}
    \item Each DML operation on a table also requires corresponding updates to its indexes.
    \item Therefore, `CREATE INDEX` statements are usually used only when the expected performance gain for reads outweighs the performance loss for writes.
\end{itemize}

\section*{Index Selection}
Deciding what indexes to build on a given database is a crucial task, often part of a Database Administrator's (DBA) job.
\begin{itemize}
    \item This is a formalized problem called "index selection," with extensive research and semi-automated tuning products.
    \item Index selection is typically decided based on query workload patterns.
\end{itemize}

\end{document}
