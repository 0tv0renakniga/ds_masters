\documentclass{article}
\usepackage{amsmath}
\usepackage{amssymb}
\usepackage{amsthm}
\usepackage{enumitem}
\usepackage{geometry}
\geometry{a4paper, margin=1in}

% Package for code listings
\usepackage{listings}
\usepackage{xcolor}

% Configure SQL listing style
\lstdefinelanguage{SQL}{
  keywords={SELECT, FROM, WHERE, COUNT, AS, GROUP BY, ORDER BY, DESC, EXPLAIN, ANALYZE, SUM},
  keywordstyle=\color{blue}\bfseries,
  identifierstyle=\color{black},
  stringstyle=\color{red},
  commentstyle=\color{green!60!black}\itshape,
  showstringspaces=false,
  tabsize=2,
  breaklines=true,
  columns=fullflexible,
  upquote=true,
  basicstyle=\ttfamily\small
}
\lstset{language=SQL}

\title{RDBMS, Query Plans and Evaluation: Comprehensive Review}
\author{DSC 208R - Data Management for Analytics}
\date{June 2025}

\begin{document}

\maketitle

\section*{Overview of RDBMS}
A Relational Database Management System (RDBMS) is a software system that implements the relational model, enabling users to store and process relational databases. RDBMS software is a significant industry, with many popular open-source options also available. Examples of RDBMS include ORACLE, Microsoft SQL Server, Amazon Redshift, Google Big Query, SAP, Snowflake, Pivotal, HP, PostgreSQL, MySQL, SQLite, and DuckDB.

\subsection*{Life of an SQL Query}
The process of an SQL query within an RDBMS involves several stages:
\begin{enumerate}
    \item A user submits an SQL Query (e.g., via Web Forms, Application Front Ends, or SQL Interface).
    \item The query is sent to the Database Server.
    \item \textbf{Parser}: Parses the SQL query to check for syntax correctness and converts it into an internal representation, often a syntax tree or Logical Query Plan (LQP).
    \item \textbf{Optimizer}: Takes the LQP and generates an optimized Physical Query Plan (PQP). It considers various execution strategies to find the most efficient one.
    \item \textbf{Query Scheduler}: Manages the execution of the PQP, potentially breaking it into segments for parallel or distributed execution.
    \item \textbf{Execute Operators (Plan Executor / Operator Evaluator)}: Executes the physical operations defined in the PQP. This interacts with various DBMS components like Files and Access Methods, Buffer Manager, and Disk Space Manager.
    \item \textbf{Result}: The query result is returned to the user.
\end{enumerate}

Other core components of an RDBMS include:
\begin{itemize}
    \item \textbf{Transaction Manager}: Manages transactions, ensuring atomicity, consistency, isolation, and durability (ACID properties).
    \item \textbf{Recovery Manager}: Handles system failures and ensures the database can be restored to a consistent state.
    \item \textbf{Lock Manager / Concurrency Control}: Manages concurrent access to data to prevent conflicts.
    \item \textbf{System Catalog}: Stores metadata about the database schema (tables, columns, indexes, etc.).
    \item \textbf{Index Files and Data Files}: Where the actual data and indexes are stored on disk.
\end{itemize}

\section*{Query Plans and Evaluation}
Query plans are structured representations of how an RDBMS will execute an SQL query.

\subsection*{Logical Query Plan (LQP)}
\begin{itemize}
    \item A Directed Acyclic Graph (DAG) with vertices representing "Logical Operators" from Extended Relational Algebra.
    \item It tells *what* is computed.
    \item Each logical operator can have alternate "physical" implementations.
\end{itemize}

\subsection*{Physical Query Plan (PQP)}
\begin{itemize}
    \item A DAG with vertices called "Physical Operators".
    \item It specifies the exact algorithm/code to run for each logical operation, including all parameters.
    \item It tells *how* the query is computed.
    \item A single logical query can have many possible physical plans.
\end{itemize}

\textbf{Example SQL Query (Netflix Schema):} 
\begin{lstlisting}[language=SQL]
SELECT M.Year, COUNT(*) AS NumBest
FROM Ratings R, Movies M
WHERE R.MID = M.MID
AND R.Stars = 5
GROUP BY M.Year
ORDER BY NumBest DESC;
\end{lstlisting}
\textbf{Netflix Schema:}
\begin{itemize}
    \item \textbf{Ratings} (RatingID, Stars, RateDate, UID, MID) 
    \item \textbf{Users} (UID, Name, Age, Join Date) 
    \item \textbf{Movies} (MID, Name, Year, Director) 
\end{itemize}

\textbf{Logical Query Plan for the example query:} 
\begin{verbatim}
Result Table
   ^
   |
  SORT (On NumBest)
   ^
   |
  GROUP BY AGGREGATE (M.Year, COUNT(*))
   ^
   |
  JOIN (R.MID=M.MID)
  /    \
 SELECT  SELECT
 R.stars=5 No predicate
  /        \
Ratings Table  Movies Table
\end{verbatim}

\textbf{Physical Query Plan for the example query (assuming B+ Tree Index on Ratings.Stars):} 
\begin{verbatim}
Result Table
   ^
   |
  External Merge-Sort (In-mem quicksort; B=50)
   ^
   |
  Hash-based Aggregate
   ^
   |
  Index-Nested Loop Join
  /                  \
 Indexed Access     File Scan
 Use Index on Stars  Read heapfile
  /                  \
Ratings Table       Movies Table
\end{verbatim}

\section*{RDBMS Logical-Physical Separation}
The concept of Logical-Physical Separation (or data independence) is a hallmark of RDBMSs.
\begin{itemize}
    \item \textbf{Benefits}:
        \begin{itemize}
            \item Increased user productivity.
            \item Automated optimization leads to faster and more scalable code.
            \item Application portability, as internal system changes do not affect the application.
        \end{itemize}
    \item This declarativity has influenced other systems like Hadoop/MapReduce, data visualization, graph processing systems, and scalable ML systems.
\end{itemize}

\section*{Indexes and Their Usage}
Indexes are data structures that improve the speed of data retrieval operations on a database table. They are crucial for efficient physical query plans.
\begin{itemize}
    \item Physical operators are chosen based on data and system parameters, as their runtimes can vary significantly.
    \item Common physical operators for major extended Relational Algebra (RA) logical operators include:
        \begin{itemize}
            \item \textbf{Selection}: Filescan, Index-based.
            \item \textbf{Join}: Nested loop, Hash join, Sort-merge join.
            \item \textbf{Aggregation}: Hash-based, Sort-based, Index-based.
        \end{itemize}
\end{itemize}

\section*{Glimpse of Query Optimization}
Query optimization is the process of choosing the most efficient execution plan for an SQL query.

\subsection*{`EXPLAIN [ANALYZE]` in SQL}
Most RDBMSs allow users to see the logical and physical plans generated for a query using the `EXPLAIN` command, sometimes with runtimes attached (with `ANALYZE`).
\begin{itemize}
    \item This command can offer insights when a query runs too slowly.
\end{itemize}

\textbf{Example `EXPLAIN` Output:} 
\begin{lstlisting}
EXPLAIN SELECT sum(i) FROM foo WHERE i < 10;

                        QUERY PLAN
------------------------------------------------------
 Aggregate (cost=23.93..23.93 rows=1 width=4)
   -> Index Scan using fi on foo (cost=0.00..23.92 rows=6 width=4)
        Index Cond: (i < 10)
(3 rows)
\end{lstlisting}
This output shows that the query will perform an `Index Scan` on `foo` using index `fi` with the condition `i < 10`, and then an `Aggregate` operation. 

\end{document}
