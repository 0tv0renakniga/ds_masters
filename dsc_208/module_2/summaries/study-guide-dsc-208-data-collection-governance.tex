\documentclass[12pt]{article}
\usepackage[utf8]{inputenc}
\usepackage{amsmath}
\usepackage{amssymb}
\usepackage{graphicx}
\usepackage{enumitem}
\usepackage{hyperref}
\usepackage{xcolor}
\usepackage{tcolorbox}
\usepackage{fancyhdr}

\hypersetup{
    colorlinks=true,
    linkcolor=blue,
    filecolor=magenta,      
    urlcolor=cyan,
    pdftitle={Study Guide: Data Collection and Governance},
    pdfpagemode=FullScreen,
}

\pagestyle{fancy}
\fancyhf{}
\rhead{DSC 208R}
\lhead{Data Collection and Governance}
\cfoot{\thepage}

\title{Study Guide: Data Collection and Governance}
\author{DSC 208R - Data Management for Analytics}
\date{}

\begin{document}

\maketitle

\begin{tcolorbox}[colback=blue!5!white,colframe=blue!75!black,title=Overview]
This study guide covers the sourcing stage of the data science lifecycle, focusing on data acquisition, reorganization, preparation, and labeling. It also addresses principles of data governance and privacy laws that every data scientist should understand.
\end{tcolorbox}

\section{Learning Objectives}

By the end of this module, you should be able to:

\begin{itemize}
    \item Understand the complete lifecycle of real-world data science projects
    \item Identify the time allocation challenges in data science work
    \item Explain the sourcing stage and its importance in the data science pipeline
    \item Describe the four main activities in the sourcing stage
    \item Recognize the challenges in data sourcing and preparation
    \item Understand the principles of data-centric AI
    \item Apply best practices in data governance and privacy
\end{itemize}

\section{The Data Science Lifecycle}

\subsection{Key Components}
\begin{enumerate}
    \item Data acquisition
    \item Data preparation
    \item Data cleaning
    \item Feature engineering
    \item Model selection
    \item Training \& inference
    \item Serving
    \item Monitoring
\end{enumerate}

\subsection{Time Allocation in Data Science}
\begin{tcolorbox}[colback=gray!10!white,colframe=gray!50!black,title=Research Findings]
Multiple industry surveys (CrowdFlower 2016, Kaggle 2018, IDC-Alteryx 2019) consistently show that data scientists spend the majority of their time on:
\begin{itemize}
    \item Data collection
    \item Data cleaning
    \item Data organization
\end{itemize}
Rather than on model building and algorithm development.
\end{tcolorbox}

\section{The Sourcing Stage}

\subsection{Definition}
The sourcing stage is where raw datasets are transformed into "analytics/ML-ready" datasets. This stage ends when data is prepared for:
\begin{itemize}
    \item SQL analytics for Business Intelligence
    \item Feature engineering for ML/AI analytics
\end{itemize}

\subsection{Challenges in Data Sourcing}
\begin{enumerate}
    \item \textbf{Heterogeneity}: Diverse data modalities, file formats, and sources
    \item \textbf{Access constraints}: Limited availability or permissions
    \item \textbf{Application diversity}: Various prediction applications with different requirements
    \item \textbf{Data volatility}: Unpredictable and continual edits to datasets
    \item \textbf{Data quality issues}: Messy, incomplete, ambiguous, or erroneous data
    \item \textbf{Scale}: Managing large volumes of data
    \item \textbf{Governance}: Poor data management practices in organizations
\end{enumerate}

\section{Four Key Activities in the Sourcing Stage}

\begin{tcolorbox}[colback=green!5!white,colframe=green!75!black,title=Sourcing Process Flow]
Raw Data Sources → Acquiring → Reorganizing → Cleaning → Data/Feature Engineering → Analytics Results

\textit{Note: Labeling \& Amplification may be required in some cases}
\end{tcolorbox}

\subsection{1. Data Acquisition}
\begin{itemize}
    \item Methods for obtaining data from various sources
    \item Understanding data access protocols and permissions
    \item Techniques for data extraction and collection
\end{itemize}

\subsection{2. Data Reorganization}
\begin{itemize}
    \item Transforming data into usable formats
    \item Structuring unstructured or semi-structured data
    \item Normalizing data representations
\end{itemize}

\subsection{3. Data Cleaning}
\begin{itemize}
    \item Identifying and handling missing values
    \item Detecting and correcting errors
    \item Removing duplicates and outliers
    \item Standardizing formats and units
\end{itemize}

\subsection{4. Data/Feature Engineering}
\begin{itemize}
    \item Creating new features from existing data
    \item Transforming variables for better model performance
    \item Dimensionality reduction techniques
    \item Feature selection methods
\end{itemize}

\section{Data-Centric AI}

\begin{tcolorbox}[colback=yellow!5!white,colframe=yellow!75!black,title=Data-Centric Approach]
The Data-Centric AI movement emphasizes improving data quality rather than just model architecture. This approach recognizes that high-quality, well-prepared data is often more important than sophisticated algorithms.
\end{tcolorbox}

\subsection{Principles}
\begin{itemize}
    \item Focus on systematic data improvement
    \item Iterative data refinement
    \item Consistent data labeling
    \item Comprehensive data documentation
\end{itemize}

\section{Data Governance and Privacy}

\subsection{Data Governance}
\begin{itemize}
    \item Policies for data management
    \item Data quality standards
    \item Data lifecycle management
    \item Roles and responsibilities
\end{itemize}

\subsection{Privacy Considerations}
\begin{itemize}
    \item Relevant privacy laws and regulations
    \item Anonymization and pseudonymization techniques
    \item Consent management
    \item Data minimization principles
\end{itemize}

\section{Study Questions}

\begin{enumerate}
    \item Why do data scientists spend more time on data preparation than on model building?
    \item What are the main challenges in the sourcing stage of the data science lifecycle?
    \item How does heterogeneity of data sources affect the data preparation process?
    \item Explain the relationship between data cleaning and feature engineering.
    \item What is the data-centric AI movement, and why is it significant?
    \item How do data governance policies impact the work of data scientists?
    \item What are the four main activities in the sourcing stage, and how do they relate to each other?
    \item Why is data labeling sometimes necessary, and what challenges does it present?
\end{enumerate}

\section{Additional Resources}

\begin{itemize}
    \item \href{https://datacentricai.org/}{Data-Centric AI Movement}
    \item CrowdFlower Data Science Report 2016
    \item Kaggle State of ML and Data Science Survey 2018
    \item IDC-Alteryx State of Data Science and Analytics Report 2019
\end{itemize}

\begin{tcolorbox}[colback=red!5!white,colframe=red!75!black,title=Key Takeaway]
The sourcing stage is the foundation of successful data science projects. Mastering the skills of data acquisition, reorganization, cleaning, and engineering is essential for any data scientist, as these activities consume the majority of time in real-world projects.
\end{tcolorbox}

\end{document}