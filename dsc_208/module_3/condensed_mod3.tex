
\documentclass{article}
\usepackage{amsmath}
\usepackage{amssymb}
\usepackage{graphicx}
\usepackage{listings}
\usepackage{array}

\begin{document}

\section*{SQL Fundamentals}

\subsection*{SQL Overview}
\begin{itemize}
    \item Data Definition Language (DDL)
    \item Create/alter/delete tables and their attributes
    \item Data Manipulation Language (DML)
    \item Query one or more tables insert/delete/modify tuples in tables
    \item Triggers and Advanced Constraints
    \item Actions executed by DBMS on updates and specify complex integrity constraints
\end{itemize}

\subsection*{Basic Query Structure}
\begin{verbatim}
SELECT [DISTINCT] <column expression list>
FROM <list of tables>
WHERE <predicate>
\end{verbatim}

\begin{itemize}
    \item Specifies columns to be retained in the results
    \item Specifies cross-product of tables
    \item Specifies selection conditions on the tables mentioned in the FROM clause
    \item The DISTINCT keyword ensures the resulting table does not have duplicates (optional)
\end{itemize}

\subsection*{Projection and Selection}
Projection selects specific columns, while selection filters rows based on conditions.

\textbf{Example Tables:}

Movie (name, year, genre)
\begin{tabular}{lll}
\hline
Name & Year & Genre \\
\hline
Apocalypse Now & 1979 & War \\
The God Father & 1972 & Crime \\
Planet Earth II & 2016 & Nature Documentary \\
\hline
\end{tabular}

ActedIN (actorname, moviename)
\begin{tabular}{ll}
\hline
Actorname & Moviename \\
\hline
Marlon Brando & Apocalypse Now \\
Al Pacino & The God Father \\
Marlon Brando & The God Father \\
\hline
\end{tabular}

\textbf{Projection Example:}
\begin{verbatim}
SELECT name, genre
FROM Movies
\end{verbatim}

\textbf{Result:}
\begin{tabular}{ll}
\hline
Name & Genre \\
\hline
Apocalypse Now & War \\
The God Father & Crime \\
Planet Earth II & Nature Documentary \\
\hline
\end{tabular}

\textbf{Selection Example:}
\begin{verbatim}
SELECT *
FROM Movies
WHERE year > 2000
\end{verbatim}

\textbf{Result:}
\begin{tabular}{lll}
\hline
Name & Year & Genre \\
\hline
Planet Earth II & 2016 & Nature Documentary \\
\hline
\end{tabular}

\textbf{Combined Projection and Selection:}
\begin{verbatim}
SELECT name
FROM Movies
WHERE year > 2000
\end{verbatim}

\textbf{Result:}
\begin{tabular}{l}
\hline
Name \\
\hline
Planet Earth II \\
\hline
\end{tabular}

\section*{SQL Joins}

\subsection*{Types of Joins}
\begin{itemize}
    \item Inner Join
    \item Self Join
    \item Outer Join (Left, Right, Full)
\end{itemize}

\subsection*{Inner Joins}
Inner joins combine rows from two or more tables based on a related column.

\textbf{Example:} Return all movie genres that Marlon Brando has acted in

\begin{verbatim}
SELECT DISTINCT genre
FROM Movie, ActedIN
WHERE Movie.name = ActedIN.moviename AND ActedIN.actorname = 'Marlon Brando'
\end{verbatim}

\textbf{Input Tables:}
\begin{tabular}{lll}
\hline
Name & Year & Genre \\
\hline
Apocalypse Now & 1979 & War \\
The God Father & 1972 & Crime \\
Planet Earth II & 2016 & Nature Documentary \\
\hline
\end{tabular}

\begin{tabular}{ll}
\hline
Actorname & Moviename \\
\hline
Marlon Brando & Apocalypse Now \\
Al Pacino & The God Father \\
Marlon Brando & The God Father \\
\hline
\end{tabular}

\textbf{Join Result (intermediate):}
\begin{tabular}{lllll}
\hline
Name & Year & Genre & Actorname & Moviename \\
\hline
Apocalypse Now & 1979 & War & Marlon Brando & Apocalypse Now \\
The God Father & 1972 & Crime & Marlon Brando & The God Father \\
\hline
\end{tabular}

\textbf{Final Result:}
\begin{tabular}{l}
\hline
Genre \\
\hline
War \\
Crime \\
\hline
\end{tabular}

\subsection*{Self Joins}
A self join is a regular join, but the table is joined with itself.

\textbf{Example:} Find name of employees and the name of their managers

\begin{verbatim}
SELECT e.name, m.name
FROM Employee e, Employee m
WHERE e.managerid = m.eid
\end{verbatim}

\textbf{Input Table:}
\begin{tabular}{llll}
\hline
eid & name & salary & managerid \\
\hline
101 & John & 50000 & 103 \\
102 & Alice & 60000 & 104 \\
103 & Mary & 80000 & NULL \\
104 & Bob & 80000 & 103 \\
\hline
\end{tabular}

\textbf{Result:}
\begin{tabular}{ll}
\hline
e.name & m.name \\
\hline
John & Mary \\
Alice & Bob \\
Bob & Mary \\
\hline
\end{tabular}

\subsection*{Outer Joins}
Outer joins return all rows from one or both tables, including unmatched rows.

\textbf{Left Outer Join:}
\begin{verbatim}
SELECT Movie.name, ActedIN.actorname
FROM Movie LEFT OUTER JOIN ActedIN
ON Movie.name = ActedIN.moviename
\end{verbatim}

\textbf{Result:}
\begin{tabular}{ll}
\hline
Movie.name & ActedIN.actorname \\
\hline
Apocalypse Now & Marlon Brando \\
The God Father & Al Pacino \\
The God Father & Marlon Brando \\
Planet Earth II & NULL \\
\hline
\end{tabular}

\textbf{Right Outer Join:}
\begin{verbatim}
SELECT Movie.name, ActedIN.actorname
FROM Movie RIGHT OUTER JOIN ActedIN
ON Movie.name = ActedIN.moviename
\end{verbatim}

\textbf{Result:}
\begin{tabular}{ll}
\hline
Movie.name & ActedIN.actorname \\
\hline
Apocalypse Now & Marlon Brando \\
The God Father & Al Pacino \\
The God Father & Marlon Brando \\
NULL & Leonardo DiCaprio \\
\hline
\end{tabular}

\textbf{Full Outer Join:}
\begin{verbatim}
SELECT Movie.name, ActedIN.actorname
FROM Movie FULL OUTER JOIN ActedIN
ON Movie.name = ActedIN.moviename
\end{verbatim}

\textbf{Result:}
\begin{tabular}{ll}
\hline
Movie.name & ActedIN.actorname \\
\hline
Apocalypse Now & Marlon Brando \\
The God Father & Al Pacino \\
The God Father & Marlon Brando \\
Planet Earth II & NULL \\
NULL & Leonardo DiCaprio \\
\hline
\end{tabular}

\section*{SQL Aggregation and Grouping}

\subsection*{Aggregate Functions}
Five basic aggregate operations in SQL:
\begin{itemize}
    \item COUNT: counts how many rows are in a particular column
    \item SUM: adds together all the values in a particular column
    \item MIN and MAX: return the lowest and highest values in a particular column, respectively
    \item AVG: calculates the average of a group of selected values
\end{itemize}

Except COUNT, all aggregations apply to a single attribute.

\textbf{Examples:}
\begin{verbatim}
SELECT count(*)
FROM Movie

SELECT count(DISTINCT genre)
FROM Movie

SELECT count(genre)
FROM Movie
\end{verbatim}

\subsection*{NULL Handling in Aggregates}
NULL is ignored in any aggregation (It does not contribute to any aggregate).

\textbf{Example Table:}
\begin{tabular}{llllll}
\hline
Name & Year & Genre & Budget & Revenue & Rate \\
\hline
Pirates of the Caribbean & 2007 & Action & \$300M & \$900M & 7.1 \\
The Lion King & 2019 & Animation & \$260M & \$1.65B & 6.5 \\
The Dark Knight & 2008 & Action & NULL & NULL & 9.5 \\
Toy Story 3 & NULL & Animation & \$300M & \$1B & 8.3 \\
American Sniper & 2013 & Action & \$59M & \$350M & 7.3 \\
\hline
\end{tabular}

\textbf{Examples:}
\begin{verbatim}
SELECT count(*)
FROM Movie
-- Returns 5

SELECT count(year)
FROM Movie
-- Returns 4 (ignores NULL)

SELECT sum(revenue)
FROM Movie
-- Returns sum of non-NULL revenues
\end{verbatim}

\subsection*{GROUP BY Operations}
GROUP BY groups rows that have the same values into summary rows.

\textbf{Example:} Find the total revenue for all movies produced after 2008 by genre

\begin{verbatim}
SELECT genre, SUM(revenue) AS TotalRevenue
FROM Movie
WHERE year > 2008
GROUP BY genre
\end{verbatim}

\textbf{Input Table:}
\begin{tabular}{llll}
\hline
Name & Year & Genre & Revenue \\
\hline
The Lion King & 2019 & Animation & \$1.65B \\
Toy Story 3 & 2010 & Animation & \$1B \\
American Sniper & 2013 & Action & \$350M \\
\hline
\end{tabular}

\textbf{Result:}
\begin{tabular}{ll}
\hline
Genre & TotalRevenue \\
\hline
Animation & \$2.65B \\
Action & \$350M \\
\hline
\end{tabular}

\textbf{Multiple Grouping Attributes:}
\begin{verbatim}
SELECT genre, year, SUM(revenue - budget) AS TotalProfit
FROM Movie
GROUP BY genre, year
\end{verbatim}

\textbf{Important:} Everything in SELECT must be either a GROUP BY attribute or an aggregate.

\section*{SQL Subqueries and Quantifiers}

\subsection*{Subquery Overview}
\begin{itemize}
    \item Subquery: A query that is part of another
    \item Nested Query: A query that has an embedded subquery
    \item A subquery can be a nested query itself
\end{itemize}

A subquery may occur in:
\begin{itemize}
    \item A SELECT clause
    \item A FROM clause
    \item A WHERE clause
\end{itemize}

Rule of thumb: avoid nested queries when possible (but sometimes it's impossible).

\subsection*{Subqueries in SELECT, FROM, and WHERE}

\textbf{1. Subqueries in SELECT:}

\begin{verbatim}
SELECT a.actorname, (SELECT genre
                     FROM Movie m
                     WHERE m.name = a.moviename) as genre
FROM ActedIn a
\end{verbatim}

This is a "correlated subquery" because the inner query references the outer query.

\textbf{Equivalent unnested query:}
\begin{verbatim}
SELECT a.actorname, genre
FROM ActedIn a, Movie m
WHERE m.name = a.moviename
\end{verbatim}

\textbf{2. Subqueries in FROM:}

\begin{verbatim}
SELECT x.name, rating
FROM (SELECT *
      FROM Movie AS m
      WHERE rating > 8) as x
WHERE x.rating < 9
\end{verbatim}

\textbf{Alternative using WITH:}
\begin{verbatim}
WITH myTable AS (SELECT * FROM Movie AS m WHERE rating > 8)
SELECT x.name, x.rating
FROM myTable as x
WHERE x.rating < 9
\end{verbatim}

\textbf{3. Subqueries in WHERE:}

\textbf{Existential Quantifiers:}

Find the name of actors who have acted in some Sci-Fi movie:
\begin{verbatim}
SELECT DISTINCT a.actorname
FROM ActedIn a
WHERE EXISTS (SELECT m.name
              FROM Movie m
              WHERE m.name = a.moviename AND
                    m.genre = 'Sci-Fi')
\end{verbatim}

Using IN:
\begin{verbatim}
SELECT DISTINCT a.actorname
FROM ActedIn a
WHERE a.moviename IN (SELECT m.name
                      FROM Movie m
                      WHERE m.name = a.moviename AND
                            m.genre = 'Sci-Fi')
\end{verbatim}

\textbf{Universal Quantifiers:}

Retrieve all actor names that only acted on action movies:
\begin{verbatim}
SELECT DISTINCT a.actorname
FROM ActedIn a
WHERE a.actorname NOT IN (SELECT a.actorname
                         FROM Movie m, ActedIn a
                         WHERE m.name = a.moviename AND
                               m.genre != 'Action')
\end{verbatim}

\textbf{Numeric Comparisons:}

Retrieve all actor names that acted in at most two action movies:
\begin{verbatim}
SELECT DISTINCT a.actorname
FROM ActedIn a
WHERE 2 >= (SELECT count(*)
            FROM Movie m
            WHERE m.name = a.moviename AND
                  m.genre = 'Action')
\end{verbatim}

\section*{Relational Data Model}

\subsection*{Data Models Overview}
A data model is an abstraction for describing and representing data. The description consists of three parts:
\begin{itemize}
    \item Structure
    \item Constraints
    \item Manipulation
\end{itemize}

\textbf{Important Data Models:}
\begin{itemize}
    \item Relational: Data represented as a collection of tables (Most Database Systems)
    \item Semistructured: Data represented as a tree
    \item Key-value pairs: Data represented as a dictionary or Hash table (NoSQL database systems)
    \item Graph
    \item Array/Matrix (Machine Learning)
    \item Dataframes
\end{itemize}

\subsection*{Relational Structure}
\begin{itemize}
    \item Data is a collection of relations
    \item A relation is a table that consists of a set of tuples or records
    \item Attribute (Field, Column) is atomic typed data entry
\end{itemize}

\textbf{Example Relation:}
\begin{tabular}{lllll}
\hline
SID & Name & Surname & Age & GPA \\
\hline
1 & Alicia & Shan & 20 & 3.5 \\
2 & Andre & Lorde & 21 & 3 \\
3 & Yan & Ke & 19 & 4 \\
4 & Sudip & Roy & 22 & 4 \\
\hline
\end{tabular}

\textbf{Relational Schema:}\\
Describes the relation's name, attribute name, and their domain name (metadata)

Student (sid: string, name: string, surname: string, age: integer, gpa: real)

\textbf{Key Concepts:}
\begin{itemize}
    \item Tuple (Record, Row): a single entry in the table
    \item Relational Instance: a set of tuples conforming to the same schema (data)
    \item Cardinality: the number of tuples in a relation (4 in the example)
    \item Arity: the number of attributes of a relation (5 in the example)
\end{itemize}

\subsection*{Integrity Constraints and Keys}
Data is only as good as information stored in it. The relational data model allows us to impose various constraints on data.

\textbf{Integrity Constraints (IC):} conditions specified on a database schema that restrict the data that can be stored.

\textbf{Key Constraint:} a statement that a minimal subset of attributes uniquely identify a tuple.

\textbf{Types of Keys:}
\begin{itemize}
    \item (Candidate) Key: a set of attributes that uniquely identify a tuple
    \item Super Key: a set of attributes that contain a key
    \item Primary Key: a database designer identifies one key and designates it as primary key
    \item Composite Key: a key consisting of multiple attributes
\end{itemize}

\textbf{Foreign Key Constraint:}\\
Sometimes data stored in a relation is linked to data stored in another relation. If one of the relations is modified, the other should be checked for consistency.

\textbf{Example:}\\
Student (sid, name, surname, age, gpa)\\
Enrolled (cid, sid, grade)

Here, sid in Enrolled is a foreign key referencing the primary key sid in Student.

\subsection*{SQL for Data Definition and Manipulation}

\textbf{Data Definition Language (DDL):}
\begin{verbatim}
CREATE TABLE Students(
  sid CHAR(20),
  name CHAR(30),
  surname CHAR(20),
  age INTEGER
)

ALTER TABLE Student
ADD Email varchar(255)

DROP TABLE Student
\end{verbatim}

\textbf{Data Manipulation Language (DML):}
\begin{verbatim}
INSERT INTO Students (sid, name, surname, age)
VALUES ('1', 'Alicia', 'Shan', 20)

UPDATE Students
SET gpa = gpa + 0.5
WHERE name = 'Alicia'

DELETE FROM Students 
WHERE name = 'Ziaho'
\end{verbatim}

\textbf{First Normal Form (1NF):}\\
All relations must be flat: we say that the relation is in first normal form.

\textbf{Example of 1NF:}\\
Instead of storing courses as a nested structure:

\begin{tabular}{lllll}
\hline
SID & Name & Surname & Age & GPA \\
\hline
1 & Alicia & Shan & 20 & 3.5 \\
2 & Andre & Lorde & 21 & 3 \\
\hline
\end{tabular}

\begin{tabular}{lll}
\hline
CID & SID & Grade \\
\hline
dsc100 & 1 & 97 \\
dsc80 & 1 & 90 \\
dsc100 & 2 & 91 \\
\hline
\end{tabular}

\section*{DataFrame Data Model}

\subsection*{Origins and Characteristics}
\begin{itemize}
    \item 1992: Emerged in S programming language at Bell Labs
    \item 2000: Inherited by R programming language
    \item 2009: Brought to Python by Pandas
\end{itemize}

DataFrames support relational operators (e.g., filter, join), linear algebra (e.g., transpose), and spreadsheet-like (e.g., pivot) operators.

\textbf{Example DataFrame:}
\begin{tabular}{lllll}
\hline
Name & FName & City & Age & Salary \\
\hline
Smith & John & 3 & 35 & \$280 \\
Doe & Jane & 1 & 28 & \$325 \\
Brown & Scott & 3 & 41 & \$265 \\
Howard & Shemp & 4 & 48 & \$359 \\
Taylor & Tom & 2 & 22 & \$250 \\
\hline
\end{tabular}

\subsection*{Comparison with Relational Model}
\textbf{In Comparison to Relational Tables:}
\begin{itemize}
    \item Lazily-induced schema
    \item Rows are named and ordered
    \item Heterogeneous data types
\end{itemize}

\textbf{In Comparison to Matrices:}
\begin{itemize}
    \item Rows and columns are labeled
    \item Columns and rows equivalent
\end{itemize}

\section*{SQL Core Concepts}

\subsection*{Summary}
SQL (Structured Query Language) is the standard language for interacting with relational databases. It supports:
\begin{itemize}
    \item Data Definition Language (DDL): Creating, altering, and deleting tables and attributes
    \item Data Manipulation Language (DML): Inserting, updating, deleting, and querying data
    \item Querying: Using SELECT statements with projection, selection, joins, grouping, and aggregation
    \item Constraints: Enforcing data integrity via keys and foreign keys
\end{itemize}

\subsection*{Core Concept Examples}

\textbf{Basic Aggregation and GROUP BY:}
\begin{verbatim}
SELECT genre, SUM(revenue) AS TotalRevenue
FROM Movie
WHERE year > 2008
GROUP BY genre
\end{verbatim}

\textbf{HAVING Clause (Derived Example):}
\begin{verbatim}
SELECT genre, COUNT(*) AS MovieCount
FROM Movie
GROUP BY genre
HAVING COUNT(*) > 2
\end{verbatim}
This query returns genres with more than two movies.

\textbf{Multiple Aggregates with HAVING (Derived Example):}
\begin{verbatim}
SELECT genre, AVG(rating) AS AvgRating, SUM(revenue) AS TotalRevenue
FROM Movie
GROUP BY genre
HAVING AVG(rating) > 8.0
\end{verbatim}
Returns genres where the average rating is above 8.0.

\subsection*{Key Points}
\begin{itemize}
    \item SQL queries can project, filter, join, group, and aggregate data
    \item GROUP BY is used to aggregate data by one or more columns
    \item HAVING filters groups after aggregation
    \item Aggregates ignore NULL values
    \item All columns in SELECT must be either grouped or aggregated
\end{itemize}

\section*{DataFrame Core Concepts}

\subsection*{Summary}
DataFrames are a tabular data structure supporting relational, linear algebra, and spreadsheet-like operations. They are widely used in data science and analytics, with origins in S, R, and Python's pandas library.

\subsection*{Core Concept Examples}

\textbf{Basic DataFrame Operations (Derived Example):}
\begin{verbatim}
# Filtering rows where Age > 30
df_filtered = df[df['Age'] > 30]

# Grouping and aggregating
df_grouped = df.groupby('City')['Salary'].mean()
\end{verbatim}

\textbf{Advanced DataFrame Operations (Derived Example):}
\begin{verbatim}
# Pivot table: Average Salary by City and Age Group
df['AgeGroup'] = pd.cut(df['Age'], bins=[20, 30, 40, 50], 
                        labels=['20-30', '31-40', '41-50'])
pivot = df.pivot_table(values='Salary', index='City', 
                       columns='AgeGroup', aggfunc='mean')

# Merging DataFrames (similar to SQL JOIN)
merged = pd.merge(df1, df2, left_on='SID', right_on='SID', how='inner')
\end{verbatim}

\subsection*{Key Points}
\begin{itemize}
    \item DataFrames support relational (filter, join), linear algebra (transpose), and spreadsheet-like (pivot) operations
    \item Schema is often inferred from data (lazily-induced)
    \item Rows and columns are labeled and can be heterogeneous
    \item Advanced operations include pivot tables, merges (joins), and groupby-aggregate patterns
\end{itemize}

\end{document}

