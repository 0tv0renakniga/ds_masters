\documentclass[12pt]{article}
\usepackage[utf8]{inputenc}
\usepackage{amsmath}
\usepackage{amssymb}
\usepackage{graphicx}
\usepackage{enumitem}
\usepackage{hyperref}
\usepackage{xcolor}
\usepackage{tcolorbox}
\usepackage{fancyhdr}
\usepackage{array}
\usepackage{listings}

\lstset{
    basicstyle=\ttfamily\small,
    breaklines=true,
    frame=single
}

\hypersetup{
    colorlinks=true,
    linkcolor=blue,
    filecolor=magenta,      
    urlcolor=cyan,
    pdftitle={Study Guide: Data Models},
    pdfpagemode=FullScreen,
}

\pagestyle{fancy}
\fancyhf{}
\rhead{DSC 208R}
\lhead{Data Models}
\cfoot{\thepage}

\title{Study Guide: Data Models}
\author{DSC 208R - Data Management for Analytics}
\date{}

\begin{document}

\maketitle

\begin{tcolorbox}[colback=blue!5!white,colframe=blue!75!black,title=Module Overview]
This guide covers fundamental data models with focus on relational and DataFrame paradigms. Key topics include structural components, constraints, and SQL operations.
\end{tcolorbox}

\section{DataFrame Model}
\begin{tcolorbox}[colback=yellow!5!white,colframe=yellow!75!black,title=Core Concepts]
\textbf{Historical Development:}
\begin{itemize}
    \item 1992: Originated in S language (Bell Labs)
    \item 2000: Adopted by R
    \item 2009: Pandas implementation in Python
\end{itemize}

\textbf{Key Features:}
\begin{itemize}
    \item Hybrid operations: Relational + Linear Algebra + Spreadsheet
    \item Labeled axes (rows \& columns)
    \item Heterogeneous data types per column
\end{itemize}
\end{tcolorbox}

\begin{tcolorbox}[colback=green!5!white,colframe=green!75!black,title=Comparative Analysis]
\renewcommand{\arraystretch}{1.5}
\begin{tabular}{>{\bfseries}l p{5cm} p{5cm}}
  \textbf{Aspect} & \textbf{vs. Relational} & \textbf{vs. Matrices} \\ \hline
  Schema & Lazily-induced & N/A \\
  Structure & Named/ordered rows \& columns & Numeric indices \\
  Data Types & Heterogeneous columns & Homogeneous elements \\
  Operations & Filter/Join + Transpose + Pivot & Pure linear algebra \\
\end{tabular}
\end{tcolorbox}

\section{Relational Model}
\begin{tcolorbox}[colback=blue!5!white,colframe=blue!75!black,title=Structural Components]
\begin{lstlisting}[language=SQL]
CREATE TABLE Students (
    sid     CHAR(20) PRIMARY KEY,
    name    CHAR(30),
    age     INTEGER,
    gpa     REAL
);
\end{lstlisting}

\textbf{Core Elements:}
\begin{itemize}
    \item \textbf{Relation}: Table with attributes (columns) and tuples (rows)
    \item \textbf{Schema}: Structural metadata (name:type pairs)
    \item \textbf{Instance}: Current dataset conforming to schema
\end{itemize}
\end{tcolorbox}

\begin{tcolorbox}[colback=red!5!white,colframe=red!75!black,title=Constraints]
\textbf{Domain Constraints:}
\begin{itemize}
    \item Enforce data types (INT, CHAR, DATE)
\end{itemize}

\textbf{Key Constraints:}
\begin{itemize}
    \item Candidate Key: Minimal unique identifier
    \item Primary Key: Chosen main identifier
    \item Super Key: Superset containing candidate key
\end{itemize}

\textbf{Referential Integrity:}
\begin{lstlisting}[language=SQL]
CREATE TABLE Enrolled (
    sid CHAR(20) REFERENCES Students(sid),
    ...
);
\end{lstlisting}
\end{tcolorbox}

\section{SQL Fundamentals}
\begin{tcolorbox}[colback=green!5!white,colframe=green!75!black,title=Essential Operations]
\renewcommand{\arraystretch}{1.2}
\begin{tabular}{ll}
  \textbf{Operation} & \textbf{SQL Example} \\ \hline
  Create Table & \texttt{CREATE TABLE Students (...);} \\
  Insert Data & \texttt{INSERT INTO Students VALUES (...);} \\
  Delete & \texttt{DELETE FROM Students WHERE age > 30;} \\
  Update & \texttt{UPDATE Students SET gpa = 3.5 WHERE ...;} \\
  Query & \texttt{SELECT name, gpa FROM Students WHERE ...;} \\
  Alter & \texttt{ALTER TABLE Students ADD email VARCHAR;} \\
\end{tabular}
\end{tcolorbox}

\begin{tcolorbox}[colback=yellow!5!white,colframe=yellow!75!black,title=First Normal Form (1NF)]
\textbf{Requirement:} Atomic values, no nested/repeating groups

\textbf{Violation Example:}
\begin{lstlisting}[language=SQL]
CREATE TABLE BadDesign (
    sid INT,
    courses_enrolled ARRAY  -- Invalid
);
\end{lstlisting}

\textbf{1NF Solution:}
\begin{lstlisting}[language=SQL]
CREATE TABLE Enrolled (
    sid INT REFERENCES Students,
    cid CHAR(10),
    grade REAL
);
\end{lstlisting}
\end{tcolorbox}

\section{Key Takeaways}
\begin{tcolorbox}[colback=red!5!white,colframe=red!75!black,title=Essential Concepts]
\begin{enumerate}
    \item \textbf{DataFrame Model} bridges relational and numerical computing
    \item \textbf{Relational Model} requires explicit schema + constraints
    \item \textbf{1NF} ensures atomic values through flat table structures
    \item \textbf{SQL} enables declarative data definition and manipulation
\end{enumerate}
\end{tcolorbox}

\end{document}
