\documentclass{article}
\usepackage{amsmath,amssymb,geometry,enumitem}
\geometry{margin=1in}

\title{DSC 255: Machine Learning \\ Homework 7}
\author{}
\date{}

\begin{document}
\maketitle

\section*{Mathematical and conceptual exercises}

\begin{enumerate}[label=\textbf{\arabic*.}]
\item A linear predictor is used to solve a classification problem with three classes.
      The data are two–dimensional and the linear functions for each class are:
      \begin{itemize}
      \item Class 1: $w_{1} = (1,1),\; b_{1} = 0$
      \item Class 2: $w_{2} = (1,0),\; b_{2} = 1$
      \item Class 3: $w_{3} = (0,1),\; b_{3} = -1$
      \end{itemize}
      Draw the resulting decision boundary and clearly mark the region
      corresponding to each class.

\item \textbf{Plant recognizer.}
      Suppose you are building a plant recognition system that takes as
      input a photograph of a plant and outputs the name of that plant.
      Your intention is that this will be used throughout California.
      When picking a training set for this task, which of the following
      options would best satisfy the statistical learning framework?
      \begin{enumerate}[label=(\alph*)]
      \item Do a web search on \texttt{American plants} and download a subset of the images you find.
      \item Obtain a collection of photos from a region of the US whose flora is similar to that of California.
      \item Go to your favorite Californian city and take photos of the plants you encounter.
      \end{enumerate}
      Give a brief explanation of your answer.

\item \textbf{A shortage of data.}
      We have a binary classification problem with very high-dimensional
      data: there are a million features.  Unfortunately, we only have
      1000 training points.  Nonetheless, we train a support vector
      machine classifier and find that it works well in practice.
      What is a possible explanation for why we are able to find a good
      model despite the shortage of labeled data?

\item \textbf{Distribution shift.}
      Suppose we are building a document classification system that
      categorizes news articles according to topic: sports, politics,
      business, and so on.  To train this system, we use a corpus of
      \emph{New York Times} articles from the past decade.  However, by
      test time the distribution has changed somewhat.  Each of the
      following scenarios is an example of either \emph{covariate shift}
      or \emph{label shift}.  Say which is which, with a brief
      explanation.
      \begin{enumerate}[label=(\alph*)]
      \item There are fewer articles on sports and more on politics.
      \item The important public figures, and thus the proper nouns in the articles, have changed.
      \end{enumerate}
\end{enumerate}

\section*{Programming exercises}

For this week you will need the data file \texttt{data0.txt}, which you can
download from the course web site.

\begin{enumerate}[label=\textbf{\arabic*.}]
\item \textbf{Multiclass Perceptron.}
      Implement the multiclass Perceptron algorithm from class.
      \begin{enumerate}[label=(\alph*)]
      \item Load the data set \textit{data0.txt}.  This file contains
            2-d data in four classes (coded as 0,1,2,3).  Each row
            consists of the two coordinates of a point followed by its label.
      \item Run the multiclass Perceptron algorithm to learn a classifier.
            Create a plot that shows all data points (using different
            colors and shapes for different labels) as well as the
            decision regions.
      \end{enumerate}

\item \textbf{Multiclass SVM.}
      In this problem we will use support vector machines on the same
      data set, \texttt{data0.txt}.
      \begin{itemize}
      \item Learn a linear SVM classifier using \texttt{sklearn.svm.LinearSVC}.
            Set \texttt{loss='hinge'} and
            \texttt{multi\_class='crammer\_singer'}.  Try
            $C \in \{0.01,\;0.1,\;1.0,\;10.0\}$.
      \end{itemize}
      \begin{enumerate}[label=(\alph*)]
      \item For each value of $C$, plot the decision boundary (no need to
            show the margins).
      \item What do you notice as $C$ increases?  Briefly comment.
      \end{enumerate}
\end{enumerate}

\end{document}
